\documentclass[a4paper]{article}
\usepackage[spanish]{babel}
\usepackage[utf8]{inputenc}
\usepackage{fancyhdr}
\usepackage{charter}   % tipografia
\usepackage{graphicx}
\usepackage{makeidx}

\usepackage{float}
\usepackage{amsmath, amsthm, amssymb}
\usepackage{amsfonts}
\usepackage{sectsty}
\usepackage{wrapfig}
\usepackage{listings}
%\lstset{language=C}


\usepackage{color} % para snipets de codigo coloreados
\usepackage{fancybox}  % para el sbox de los snipets de codigo

\definecolor{litegrey}{gray}{0.94}

% \newenvironment{sidebar}{%
% 	\begin{Sbox}\begin{minipage}{.85\textwidth}}%
% 	{\end{minipage}\end{Sbox}%
% 		\begin{center}\setlength{\fboxsep}{6pt}%
% 		\shadowbox{\TheSbox}\end{center}}
% \newenvironment{warning}{%
% 	\begin{Sbox}\begin{minipage}{.85\textwidth}\sffamily\lite\small\RaggedRight}%
% 	{\end{minipage}\end{Sbox}%
% 		\begin{center}\setlength{\fboxsep}{6pt}%
% 		\colorbox{litegrey}{\TheSbox}\end{center}}

\newenvironment{codesnippet}{%
	\begin{Sbox}\begin{minipage}{\textwidth}\sffamily\small}%
	{\end{minipage}\end{Sbox}%
		\begin{center}%
		\colorbox{litegrey}{\TheSbox}\end{center}}



\input{page.layout}
%\setcounter{secnumdepth}{2}
\usepackage{underscore}
\usepackage{caratula}
\usepackage{url}

\usepackage{color}
\usepackage{clrscode3e} % para el pseudocodigo




\begin{document}


\thispagestyle{empty}
\materia{Algoritmos y Estructuras de Datos III}
\submateria{Segundo Cuatrimestre de 2014}
\titulo{Trabajo Práctico I}
\subtitulo{Problemas de optimización}
\integrante{González Alba, Pablo}{476/10}{pablo.gonzalez.alba@gmail.com}
\integrante{Quiroz, Nicol\'as}{450/11}{nquiroz@dc.uba.ar}
\integrante{Vaghi, Agustín}{790/07}{vaghiagustin@gmail.com}
\integrante{Vuotto, Lucas}{385/12}{lvuotto@dc.uba.ar}

\maketitle
\newpage

\thispagestyle{empty}
\vfill
\begin{abstract}
    \textcolor{red}{\textbf{completar!}}
\end{abstract}

\thispagestyle{empty}
\vspace{3cm}
\tableofcontents
\newpage


%\normalsize
\newpage

\section{Objetivos generales}
  \textcolor{red}{\textbf{completar!}}

\newpage

\section{Plataforma de pruebas}
El testeo de los algoritmos implementados fue realizado, principalmente, en las máquinas del laboratorio 3 del DC. \newline
\begin{itemize}
  \item \textbf{Sistema Operativo:} Ubuntu Linux 12.04 x86_64, kernel 3.2.0-30-generic
  
  \item \textbf{Especificaciones del Software:} el código está implementado en C++. Utilizamos Bash y Ruby para los scripts.
  
  \item \textbf{Especificaciones del Hardware:} Intel(R) Core(TM) i5-2500K CPU @ 3.30GHz, 8GB de RAM.
\end{itemize}

\newpage

\section{Problema 1: Puentes sobre lava caliente}
\subsection{Descripción del problema.}

\vspace*{0.3cm}

El problema consiste en encontrar una secuencia de vuelos que permita, dadas una
ciudad de origen y otra de destino y un listado de vuelos, viajar del origen al destino
para llegar lo más temprano posible.

Esta secuencia debe empezar con un vuelo que salga de la ciudad origen, luego
cada vuelo debe partir de la ciudad de llegada del anterior, dejando un intervalo
de, al menos, 2 horas. Para finalmente terminar en la ciudad destino.
Y de todas las posibles combinaciones válidas de vuelos, ser la que llegue antes
al destino.

También se deberá informar cuando el recorrido es imposible.

\vspace*{0.5cm}

\textbf{Ejemplos:}
%\begin{itemize}
  \textcolor{red}{\textbf{completar!}}
%\end{itemize}



\newpage
\subsection{Desarrollo de la idea y pseudocódigo.}

\vspace*{0.3cm}

\textcolor{red}{\textbf{completar!}}


Para resolver el problema dado, se propone ordenar todos los vuelos disponibles
en base al horario de llegada en forma ascendente. Una vez realizado el ordenamiento,
se recorre cada vuelo y se pregunta si hay algún modo de
\begin{itemize}
  \item llegar a la ciudad de origen 2 horas antes del horario de partida
  \item la ciudad de partida de dicho vuelo es la ciudad de origen del cliente
\end{itemize}
De valer alguno de los dos, si no hay otro vuelo que llegue antes a la ciudad del
destino del vuelo, se procede a marcar dicho vuelo como el primero en llegar a su
ciudad de destino y se marca como su predecesor al primero que llega a la ciudad
de origen. Caso contrario, se ignora el vuelo.

Una vez recorridos todos, si hay algún vuelo que llega a la ciudad de destino,
existe solución, la cual se construye a partir de recorrer los vuelos predecesores.
Dicha cadena tiene como primer vuelo uno que parte desde la ciudad de origen del
cliente.


\begin{codebox}
\Procname{$\proc{plan_de_vuelo}(vuelos, origen, destino)$}
\li \Comment $\id{vuelos}$ es un arreglo de tipo Vuelo, donde Vuelo es una estructura
    conformada por una ciudad origen, una ciudad destino, un tiempo de partida y
    un tiempo de arribo.
\li \Comment $\id{rutas}$ es un diccionario de ciudades en vuelos, en el cual para
    cada ciudad nos dice cual es el vuelo que llega antes a dicha ciudad.
\li $\proc{ordenar}(vuelos)$
\li $\id{rutas} \gets \empty$
\li \While $\neg \proc{vacío}(vuelos)$ \Do
\li   $\id{vuelo} \gets \proc{primero}(vuelos)$
\li   \If $\nneg \proc{existe}(\proc{destino}(vuelo), rutas) \land
          \proc{puede_tomar}(vuelo, rutas, origen)$ \Then
\li     $\id{rutas[\proc{destino}(vuelo)]} \gets vuelo$
      \End
\li   $\id{vuelos} \gets \id{vuelos} \setminus \{vuelo\}$
    \End
\li \If $\proc{existe}(destino, rutas)$ \Then
\li   \Return $\proc{armar_pila}(rutas, destino)$
\li \Else
\li   \Return $\emptyset$
    \End
\end{codebox}

\begin{codebox}
\Procname{$\proc{puede_tomar}(vuelo, rutas, origen)$}
\li \Return $\proc{origen}(vuelo) \isequal origen \lor
             (\proc{existe}(\proc{origen}(vuelo), rutas) \land
              \proc{llegada}(rutas[\proc{origen}(vuelo)]) <=
              \proc{salida}(vuelo) - 2)$
\end{codebox}

\begin{codebox}
\Procname{$\proc{armar_pila}(rutas, destino)$}
\li $\id{lista} \gets \emptyset$
\If \proc{existe}(destino, rutas)
vuelo = rutas[destino]
agregarAdelante(vuelo, lista)
while (vuelo.predecesor)
  vuelo = vuelo.predecesor
  agregarAdelante(vuelo, lista)
end
\Else
\li \Return $\id{lista}$
\End
\end{codebox}

\newpage
\subsection{Justificación de la resolución y demostración de correctitud.}

\vspace*{0.3cm}

\textcolor{red}{\textbf{completar!}}



\newpage
\subsection{Análisis de complejidad.}

\vspace*{0.3cm}

\textcolor{red}{\textbf{completar!}}



\newpage
\subsection{Experimentación y gráficos.}

\vspace*{0.3cm}

\subsubsection{Test 1 - benchmark caso aleatorio}

\textcolor{red}{\textbf{completar!}}


\newpage
\subsubsection{Test 2 - benchmark del peor caso}

\textcolor{red}{\textbf{completar!}}


\newpage
\subsubsection{Test 3 - benchmark del mejor caso}

\textcolor{red}{\textbf{completar!}}


\newpage

\section{Problema 2: Horizontes lejanos}
\subsection{Descripción del problema.}

\vspace*{0.3cm}

Dado un tablero de ajedrez de tamaño $n \times n$ y $k$ caballos ocupando
inicialmente ciertos casilleros del mismo, el objetivo del problema consiste
en reunir a todos los caballos en un mismo casillero, minimizando la
cantidad total de movimientos realizados. Esta cantidad equivale a la suma
de los movimientos de todos los caballos en el tablero para llegar a dicho
casillero.

Un caballo puede moverse únicamente respetando los movimientos válidos según
las reglas del ajedrez, pero un casillero puede estar ocupado por más de un
caballo simultáneamente.

\vspace*{0.5cm}

\textbf{Ejemplo:}

En un tablero de 8x8, con 3 caballos en las posiciones [2,2], [5,5] y [2,8],
la menor cantidad de saltos posibles es 4, haciendo que los caballos de los
extremos cayan hacia la posición del caballo del medio[5,5], como se puede ver
en la siguiente imagen:

\begin{figure}[htb]
  \begin{center}
      \includegraphics[scale=0.25]{imagenes/caballos.jpg}
  \end{center}
  \caption{ejemplo de tablero.}
\end{figure}


\newpage
\subsection{Desarrollo de la idea y pseudocódigo.}

\vspace*{0.3cm}

Para resolver este problema, utilizaremos $k$ tableros de $n \times n$
casilleros, uno por cada caballo. En cada tablero se calculará el costo para
dicho caballo de llegar a cada casillero, aplicando \textit{BFS} desde el
casillero inicial, quedando inválidos los casilleros que no pueden
alcanzarse.

Luego se recorren todos los casilleros, sumando el valor de estos en todos
los tableros (si son alcanzables), obteniendo así el costo de cada casillero
para cada caballo. De existir, el mínimo de estos valores será el casillero
que pueden alcanzar todos los caballos en la menor cantidad de saltos.

\begin{codebox}
\Procname{$\proc{puntoDeEncuentro}(caballos, n)$}
\li $\id{tableros} \gets \emptyset$
\li \For $caballo \in caballos$ \Do
\li   $\proc{agregar}(tableros,
                      \proc{llenarTablero}(\proc{crearTablero}(n),
                                           caballo))$
    \End
\li $\id{i} \gets 0$
\li $\id{j} \gets 0$
\li $\id{min_i} \gets \infty$
\li $\id{min_j} \gets \infty$
\li $\id{min} \gets \infty$
\li \While $\id{i} < \id{n}$ \Do
\li   \While $\id{j} < \id{n}$ \Do
\li     $\id{sum} \gets 0$
\li     $\id{caballo} \gets 0$
\li     \For $tablero \in tableros$ \Do
% \li       \If $tablero_{ij} \isequal \infty$ \Then
% \li         $\id{sum} \gets \infty$
% \li       \ElseIf $sum \neq \infty$ \Then
\li         $\id{sum} \gets \id{sum} + tablero_{ij}$
%           \End
        \End
\li     \If $sum < min$ \Then
\li       $\id{min} \gets \id{sum}$
\li       $\id{min_i} \gets i$
\li       $\id{min_j} \gets j$
        \End
\li   $\id{j} \gets \id{j} + 1$
      \End
\li $\id{i} \gets \id{i} + 1$
    \End
\li \Return $(\id{min}, \id{min_i}, \id{min_j})$
\end{codebox}


\begin{codebox}
\Procname{$\proc{crearTablero}(n)$}
\li \Return matriz de $n \times n$ inicializada en $\infty$
\end{codebox}


\begin{codebox}
\Procname{$\proc{llenarTablero}(tablero, inicio)$}
\li $\id{posiciones} \gets \emptyset$
\li $\proc{encolar}(posiciones, (inicio, 0))$
\li \While $\lnot \proc{vacio?}(posiciones)$ \Do
\li   $\id{pos} \gets \proc{primero}(\proc{frente}(posiciones))$
\li   $\id{nivel} \gets \proc{segundo}(\proc{frente}(posiciones))$
\li   $\proc{desencolar}(posiciones)$
\li   $\id{i} \gets inicio_i$
\li   $\id{j} \gets inicio_j$
\li   \If $tablero_{ij} \isequal \infty$ \Then
\li     $\id{tablero_{ij}} = nivel$
\li     \For $\id{v} \in \proc{vecinos}(tablero, posicion)$ \Do
\li       $\proc{encolar}(posiciones, (v, nivel + 1))$
        \End
      \End
    \End
\end{codebox}

\newpage
\subsection{Justificación de la resolución y demostración de correctitud.}

\vspace*{0.3cm}

La solución es el mínimo número de movimientos entre todos los caballos que
los deja en el mismo casillero, y la posición de ese casillero. Es decir, si
$s \in [1, \dots, n] \times [1, \dots, n]$ es solución, $s$ pertenece a
\begin{align*}
\min_{(i, j) \in [1, \dots, n] \times [1, \dots, n]} \sum_{c \in caballos}
\text{camino\_mínimo}(c, (i, j))
\end{align*}

Primero completamos, en un tablero para cada caballo, los caminos mínimos
desde la posición inicial de este hasta cada casillero que puede alcanzar.
Para esto, usamos el tablero como un grafo, considerando cada casillero como
un nodo y los posibles saltos de caballos como aristas. Empezando desde la
posición del caballo, se recorren todos los nodos alcanzables mediante el
algoritmo BFS, logrando de esta manera poner la cantidad mínima de saltos
para cada casillero alcanzable, pues BFS obtiene los caminos mínimos desde
un nodo inicial a cualquier otro nodo del grafo, siempre y cuando estos se
encuentren en la misma componente conexa.

Luego de obtener este resultado, para obtener el valor mínimo y la
coordenada donde se alcanza, se recorren todos los casilleros y se calcula
la sumatoria de costos de ese casillero en el tablero correspondiente a cada
caballo. En cada caso se compara el valor obtenido con el mínimo actual y se
actualiza este valor y la coordenada, de ser necesario.

El valor obtenido, si existe, es una solución óptima que minimiza el número
de movimientos de todos los caballos para llegar a ese casillero, pues el
algoritmo es exhaustivo: recorre todos los posibles resultados y se queda
con el menor.

\newpage
\subsection{Análisis de complejidad.}

\vspace*{0.3cm}

\textbf{Aclaraciones} : Se considerará $k$ = cantidad de caballos
 y $n$ = cantidad de casilleros de alto o largo (es indiferente ya que los
 tableros serán cuadrados).

\begin{enumerate}

 \item Las operaciones sobre el contenedor \verb|vector| de la STL (push_back, 
 begin y end) y la creación de sus iteradores toman $O(1)$.
 
 \item En la función \verb|main| la estructura \verb|for| es ejecutada
  $k$ veces, donde se realiza un \verb|make_pair| y un \verb|push_back| 
  sobre un \verb|vector| que cuestan $O(1)$, dando una complejidad de $O(k)$.
   
 \item Crear la esctructura \verb|pair| $respuesta$ que contrendrá el 
 resultado del problema, si lo hay, cuesta $O(1)$.
 
 \item En la función \verb|punto_de_encuentro| se crea el \verb|vector| 
 $tableros$ ( $O(1)$ ), y con un iterador se recorre el \verb|vector| $caballos$ en 
 $O(k)$ donde por cada iteración se crea un tablero (ver \textbf{Tablero}) 
 en $O(n^{2})$, se crea un \verb|queue| para los casilleros ( $O(1)$ ), 
 se crea un \verb|make_pair| ( $O(1)$ ) y se los inserta en el \verb|queue| $casilleros$
 mediante \verb|push| ($ O(1)$ ), se llena el tablero creado (ver \textbf{LLenar tablero}) 
 en $O(n^{2})$ y se los agrega a $tableros$ con \verb|push_back| en $O(1)$
  dando una complejidad total de $O(k.(n^{2} + n^{2}))$ = $O(k.n^{2})$.
 
 \item \textbf{Tablero} - Crear un tablero implica crear un \verb|vector| 
 que contendrá otro \verb|vector| dentro ($O(1)$) y se le dará tamaño $n$
 al vector contenedor con \verb|resize| en $O(n)$. Luego se creará un iterador
 ($O(1)$) para recorrer el \verb|vector| contenedor ($O(n)$) y en cada iteración
 al \verb|vector| contenido en dicha posición se le seteará $n$ posiciones
 en $-1$ ($O(n)$), dando una complejidad total de $O(n^{2})$.
 
 \item \textbf{Llenar tablero} - La función posee un \verb|while| donde se ejecutan
 todas las instrucciones, mientras el \verb|queue| $casilleros$ sea diferente de vacío.
 $casilleros$ comienza con 1 elemento y en cada ejecución se le realiza
 \verb|pop| de un elemento y se le agregan todas las posiciones a las cuales se 
 pueden saltar (siguendo el patron del caballo y sin que salga del tablero), la 
 complejidad del \verb|while| sera de $O(n^{2})$ en el peor caso (todos los casilleros), multiplicado 
 por la complejidad de las instrucciones que se ejecuten dentro del mismo.
 
 Crear un \verb|pair| y asignarle el \verb|first| de la tupla del primer elemento 
 de $casilleros$ cuesta $O(1)$, al igual que crear un \verb|int| y asignarle el
 \verb|second| de la tupla del primer elemento de $casilleros$.
 
 Hacer \verb|pop| de $casilleros$ cuesta $O(1)$.
 
 En el primer \verb|if| acceder al elemento $[x][y]$ del \verb|vector[vector]| 
 $t.casilleros$ del tablero y compararlo con -1 cuesta $O(1)$ (notar que $tablero$ 
 posee un $casilleros$ \verb|vector[vector]| y que la función $llenar\_tablero$ recibe otro $casilleros$ 
 \verb|queue|). En caso de ser verdadero se realiza un \verb|continue| en $O(1)$.
 
 Asignarle al \verb|int| $nivel$ un elemento de $t.casilleros$ en $[x][y]$ cuesta $O(1)$.
 
 Para las ocho posiciones posibles a las que puede saltar un caballo desde la posición
 actual, validar que la posición sea válida (que caiga dentro del tablero) cuesta 
 $O(1)$, y de ser válida ver si el valor almacenado en dicho lugar es distinto de -1
 también cuesta $O(1)$. Si se cumplen estas condiciones se ejecuta la instrucción 
 del if donde se agrega el casillero que se acaba de evaluar al \verb|queue| $casilleros$
 ($O(1)$) que se recorre en el \verb|while|, realizando dos \verb|make_pair| ($O(1)$) 
 y asignandole el valor del $nivel$ + 1 ($O(1)$).
 
\end{enumerate}

\newpage
\subsection{Experimentación y gráficos.}

\vspace*{0.3cm}

\subsubsection{Test 1 - benchmark caso aleatorio}

\textcolor{red}{\textbf{completar!}}


\newpage
\subsubsection{Test 2 - benchmark del peor caso}

Este ejercicio no tiene mejor y peor caso, todos tardan lo mismo.
Porque en todos los casos se generan los tableros para todos los caballos
y también se busca el mínimo en todos los casilleros.-
\textcolor{red}{\textbf{completar!}}


\newpage
\subsubsection{Test 3 - benchmark del mejor caso}

\textcolor{red}{\textbf{completar!}}


\newpage

\section{Problema 3: Biohazard}
\subsection{Descripción del problema.}

\vspace*{0.3cm}

En este problema, un cliente busca ofrecer un servicio particular sobre una
red existente de computadoras. Nuestra tarea consiste en elegir algunas de las
mismas, para utilizarlas como servidores formando un \textit{backbone} que 
siga una tipología de red de tipo \textit{anillo}. Además, se requiere que 
las demás computadoras de la red estén conectadas a algún servidor 
perteneciente a este anillo, para poder tener acceso al mismo.

La red cuenta con conexiones entre algunos pares de equipos, pero 
\textbf{por cada conexión utilizada se deberá pagar cierto costo 
asociado al enlace utilizado}.

\medskip

El objetivo es elegir el anillo de servidores (indicando qué servidores y
enlaces lo representarán) y todas las demás conexiones necesarias, de 
manera tal que las demás computadoras queden conectadas al anillo y el costo
asociado al uso de estos enlaces sea \textbf{mínimo}.

\vspace*{0.5cm}

\textbf{Ejemplos:}
%\begin{itemize}
  \textcolor{red}{\textbf{completar!}}
%\end{itemize}



\newpage
\subsection{Desarrollo de la idea y pseudocódigo.}

\vspace*{0.3cm}

En este problema, \textbf{trataremos a la red de computadoras como un grafo}, 
donde \textbf{las computadoras serán los vértices, las conexiones las aristas 
y el peso de cada arista será el costo de hacer dicha conexión}.

Comenzaremos por corroborar que exista una solución. Para esto \textbf{el  
grafo deberá ser conexo} y para cumplir con esto, debe tener \textbf{al 
menos tantas aristas como vértices}. Si no fuese conexo, no se podrían 
unir todas las computadoras. \textbf{Si tiene menos aristas que vértices}, 
el grafo sería un árbol y \textbf{nunca podría formarse un anillo}.

Suponiendo que tiene solución, se procederá a calcular el \textbf{árbol 
generador mínimo} mediante el \textbf{algoritmo de Prim}. Para obtener el 
anillo, \textbf{formaremos un ciclo agregando la arista de menor peso} no 
incluída en el \textit{AGM}.


\begin{codebox}
\Procname{$\proc{anillar}(G)$}
\li \If $\proc{noTieneSolucion}(G)$
\li     \Then
            \Return $\const{false}$
        \End
\li  \Return $\proc{completarAnillo}(\proc{Prim}(G,w,r))$
\end{codebox}


\begin{codebox}
\Procname{$\proc{noTieneSolucion}(G)$}
\li \Return $\proc{esConexo}(G,v) \land
    \proc{tamanio}(\proc{nodos}(G)) \leq 
    \proc{tamanio}(\proc{vertices}(G))$
\end{codebox}


\begin{codebox}
\Procname{$\proc{Prim}(G,w,r)$}
\li \Comment u: nodo cualquiera de G
\li \For $each \id{u} \in \id{G.V}$
\li     \Do
            $\id{u.clave} \gets \infty$
            $\id{u.\pi} \gets \const{nil}$
        \End
\li $\id{r.clave} \gets 0$
\li $\id{Q} \gets \id{G.V}$
\li \While $\id{Q} \neq 0$
\li     \Do
            $\id{u} \gets \proc{extraerMinimo}(Q)$
\li         \For $each \id{v} \in \id{G.Ady[u]}$
\li             \Do
\li                 \If $\id{v} \in \id{Q} \land \id{w(u,v)} < \id{v.clave}$
\li                     \Then
                            $\id{v.\pi} \gets \id{u}$
                            $\id{v.clave} \gets \id{w(u,v)}$
                        \End
                \End
        \End
\end{codebox}


\begin{codebox}
\Procname{$\proc{esConexo}(G,v)$}
\li \Comment $\id{u}$ es nodo cualquiera de G
\li \For $each \id{u} \in \id{G.V}$
\li     \Do
            $\id{u.color} \gets \const{blanco}$
            $\id{u.d} \gets \infty$
            $\id{u.\pi} \gets \const{nil}$
        \End
\li $\id{s.color} \gets \const{gris}$
\li $\id{s.d} \gets 0$
\li $\id{s.\pi} \gets \const{nil}$
\li $\id{Q} \gets \emptyset$
\li \While $\id{Q} \neq 0$
\li     \Do
            $\id{u} \gets \proc{desencolar}(Q)$
\li         \For $each \id{v} \in \id{G.Ady[u]}$
\li             \Do
\li                 \If $\id{v.color} \isequal \const{white}$
\li                     \Then
                            $\id{v.color} \gets \const{gray}$
                            $\id{v.d} \gets \id{u.d + 1}$
                            $\id{v.\pi} \gets \id{u}$
                            $\proc{encolar}(Q,v)$
                        \End
                \End
\li     $\id{u.color} \gets \const{black}$
        \End            
\end{codebox}


\begin{codebox}
\Procname{$\proc{completarAnillo}(AGM, G)$}
\li $\id{X} \gets \proc{vertices}(grafo) \setminus \proc{vertices}(AGM)$
\li $\id{m} \gets \proc{dameUno}(X)$
\li $\id{X} \gets \id{X} \setminus \{m\}$
\li \While $\neg \proc{vacio}(X)$ 
\li     \Do
\           $\id{x} \gets \proc{dame_uno}(X)$
\li         \If $\proc{peso(x)} < \proc{peso(m)}$ 
\li             \Then
                    $\id{m} \gets \id{x}$
                \End
\li         $\id{X} \gets \id{X} \setminus \{x\}$
        \End
\li \Return $\proc{agregar}(AGM, m)$
\end{codebox}



\newpage
\subsection{Justificación de la resolución y demostración de correctitud.}

\vspace*{0.3cm}

\textcolor{red}{\textbf{completar!}}



\newpage
\subsection{Análisis de complejidad.}

\vspace*{0.3cm}

\textcolor{red}{\textbf{completar!}}



\newpage
\subsection{Experimentación y gráficos.}

\vspace*{0.3cm}

\subsubsection{Test 1 - benchmark caso aleatorio}

\textcolor{red}{\textbf{completar!}}


\newpage
\subsubsection{Test 2 - benchmark del peor caso}

\textcolor{red}{\textbf{completar!}}


\newpage
\subsubsection{Test 3 - benchmark del mejor caso}

\textcolor{red}{\textbf{completar!}}


\newpage
\section{Acerca de los tests}
Los casos aleatorios se generan mediante los scripts \verb|ejN.random.rb|, donde 
N es el número correspondiente al ejercicio. Estos scripts toman siempre dos parámetros, 
siendo el primero la semilla que utilizamos para generar números pseudoaleatorios, que 
por defecto toma el valor 0 si no es especificado y el segundo parámetro corresponde 
al tamaño de la entrada. 

Además de los tests con casos aleatorios, tenemos en cuenta los mejores y peores 
casos de cada algoritmo, fijando los valores adecuados de los parámetros para 
obtenerlos.

En cuanto a la metodología para medir el tiempo, utilizamos el archivo \verb|tiempo.h|, 
que define macros para contar la cantidad de ciclos de clock producidos entre dos instantes. 
Cada instancia se repite 25 veces para reducir el impacto de los \textit{outliers}, quedándonos 
con el valor mínimo en cada caso. Estos valores se vuelcan a un archivo \verb|info.n.dat|, que 
es utilizado luego para generar los gráficos mediante gnuplot.

Para correr los tests estáticos, se ejecuta \verb|make; make test| y para los 
casos aleatorios (utilizados para los gráficos), \verb|make; make plot|. 

Todos los archivos \verb|.cc| son compilados utilizando la optimización \verb|-O3|. 
  

\newpage

\section{Conclusiones}
    \textcolor{red}{\textbf{completar!}} \medskip
    escribir alguna conclusión para cerrar el tp y todos felices.

\newpage

\section{Apéndice: secciones relevantes del código}
    \textcolor{red}{\textbf{completar!}} \medskip
    en cada problema hacer referencia a la parte correspondiente de este apéndice.


\end{document}
