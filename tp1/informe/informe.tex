
\documentclass[a4paper]{article}
\usepackage[spanish]{babel}
\usepackage[utf8]{inputenc}
\usepackage{fancyhdr}
\usepackage{charter} % tipografia
%~ \usepackage{graphicx}
%~ \usepackage{makeidx}
%~ \usepackage{float}
%~ \usepackage{amsmath}
%~ \usepackage{sectsty}
%~ \usepackage{charter}
%~ \usepackage{wrapfig}
%~ \usepackage{listings}
%~ %\lstset{language=C}
%~ \usepackage{color} % para snipets de codigo coloreados
\usepackage{fancybox}  % para el sbox de los snipets de codigo

\definecolor{litegrey}{gray}{0.94}

% \newenvironment{sidebar}{%
% 	\begin{Sbox}\begin{minipage}{.85\textwidth}}%
% 	{\end{minipage}\end{Sbox}%
% 		\begin{center}\setlength{\fboxsep}{6pt}%
% 		\shadowbox{\TheSbox}\end{center}}
% \newenvironment{warning}{%
% 	\begin{Sbox}\begin{minipage}{.85\textwidth}\sffamily\lite\small\RaggedRight}%
% 	{\end{minipage}\end{Sbox}%
% 		\begin{center}\setlength{\fboxsep}{6pt}%
% 		\colorbox{litegrey}{\TheSbox}\end{center}}

\newenvironment{codesnippet}{%
	\begin{Sbox}\begin{minipage}{\textwidth}\sffamily\small}%
	{\end{minipage}\end{Sbox}%
		\begin{center}%
		\colorbox{litegrey}{\TheSbox}\end{center}}



%~ \input{page.layout}
%~ % \setcounter{secnumdepth}{2}
%~ \usepackage{underscore}
%~ \usepackage{caratula}
%~ \usepackage{url}


\begin{document}


\section{Problema 1}

1) El problema nos pide implementar un algoritmo que, de ser posible, calcule la
cantidad mínima de saltos que debe dar un concursante de una competencia para
poder cruzar un puente, el cual tiene ciertos tablones rotos. Los mismos están
marcados y deben ser evitados para lograr su supervivencia. Cada uno de los
participantes puede saltar una determinada cantidad de tablones como máximo.

El algoritmo debe decidir si dicha hazaña es posible, y de serlo, debe especificar
la cantidad de saltos a dar y a qué tablones hacerlo.

\textbf{Ejemplos}
\begin{itemize}
  \item Para un puente de 10 tablones, y un participante capaz de saltar de a 3
  tablones, con los tablones 1, 4, 6, 8 y 9 rotos, la salida podría ser saltar al
  tablón 3, luego al 5, luego al 7, luego al 10 y luego afuera del puente.
  \item Si en cambio, en el ejemplo anterior, el tablón 7 también estuviese roto,
  el algoritmo debe informarnos que no es posible realizar la hazaña.
\end{itemize}


2) Para resolver el problema, propusimos un algoritmo goloso, pues en cada iteración
buscamos saltar la mayor cantidad posible de tablones, siempre y cuando el tablón
al que vamos a saltar esté sano. Caso contrario, se saltará al tablón anterior.
Si todos los tablones anteriores hasta la posición actual están rotos, el problema
no tiene solución.

saltarPuente(n : tablones, c : saltoMaximo)
  // n cantidad de tablones
  // c cantidad de tablones maximos a saltar
  //primer tablon = 1
  //c >= 1
  
  saltos = 0
  posicionActual = 0
  
  mientras(posicionActual <= n)
    saltoActual = c
    mientras ( noEstaSano(posicionActual + saltoActual) )
      saltoActual -- 
      si saltoActual == 0 entonces no hay solucion
    
    posicionActual += saltoActual
    saltos ++

3) Próximamente.

4) Próximamente.

5) Próximamente.

6) Próximamente.


\end{document}
