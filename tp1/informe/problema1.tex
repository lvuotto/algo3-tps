\noindent
\textbf{1.1. Descripción del problema.}

\vspace*{0.3cm}

Este problema se trata de implementar un algoritmo que, de ser posible, 
\textbf{calcule la cantidad mínima de saltos} que debe dar un participante para poder cruzar un 
puente de cierto tamaño (fijo), el cual presenta \textbf{algunos de sus tablones rotos}. Estos se encuentran 
marcados y \textbf{deben ser evitados}, de otro modo, el participante fracasa en su intento y puede llegar a 
perder la vida. Cada uno de los participantes puede saltar una \textbf{determinada cantidad de tablones como máximo}. \medskip

El algoritmo debe decidir si dicha hazaña es posible, y de ser así, 
\textbf{especificar la cantidad de saltos a dar y a qué tablones hacerlo}. \medskip

Asumimos que tanto la \textbf{longitud del puente como la cantidad de saltos máximos por participante
son valores naturales}.

\vspace*{0.5cm}

\textbf{Ejemplos:}
\begin{itemize}
  \item Para un puente de 10 tablones y un participante capaz de saltar de a 3
  tablones como máximo, teniendo los tablones 1, 4, 6, 8 y 9 rotos, la salida podría ser: 
  primero saltar al tablón 3, luego al 5, luego al 7, luego al 10 y luego afuera del puente.
  \item Si en cambio, en el ejemplo anterior, el tablón 7 también estuviese roto,
  el algoritmo debe informarnos que, bajo esas condiciones, no es posible cruzar el puente.
  \item < agregar un ejemplo mas con dibujitos y cosas lindas >
\end{itemize}


\vspace*{0.75cm} \noindent



\noindent
\textbf{1.2. Desarrollo de la idea y pseudocódigo.}

\vspace*{0.3cm}

Para resolver este problema, propusimos un algoritmo de tipo \textit{greedy}, pues en cada iteración
del mismo buscamos siempre saltar \textbf{la mayor cantidad posible de tablones} (siempre y cuando el 
tablón al que vamos a saltar esté sano), caso contrario, se saltará al tablón anterior.
Si todos los tablones anteriores, incluyendo la posición actual, están rotos, el problema
\textbf{no tiene solución}.

\vspace*{0.5cm}


\begin{codebox}
\Procname{$\proc{cruzarPuente}(n,c)$}
\li \Comment n: cantidad de tablones del puente 
\li \Comment c: cantidad de tablones máximos a saltar 
\li \Comment el primer tablón es el 1
\li \If $\neg(\proc{esPosibleCruzar}(puente, c))$
\li     \Then 
            \Return $\emptyset$
        \End
\li $\id{posicionActual} \gets 0$
\li $\id{solucion} \gets \emptyset$
\li \While $\id{posicionActual} \leq \id{n}$
\li     \Do 
            $saltoActual \gets c$
        \End
\li     \While $\neg(\proc{puedeSaltar}(puente, posicionActual + saltoActual - 1))$
\li         \Do 
                $saltoActual \gets saltoActual - 1$
            \End
\li     $\id{posicionActual} \gets \id{posicionActual} + \id{saltoActual}$
\li     $\id{solucion} \gets \id{solucion} \cup \id{posicionActual}$
\li \Return $\id{solucion}$
\end{codebox}      



cruzarPuente(n : tablones, c : saltoMáximo)
  // n cantidad de tablones
  // c cantidad de tablones maximos a saltar
  // primer tablon = 1

  si no esPosibleCruzar(puente, saltoMáximo) entonces retornar $\emptyset$

  posicionActual $\gets$ 0
  solución $\gets \emptyset$

  mientras posicionActual $\leq$ n
    saltoActual $\gets$ saltoMáximo
    mientras no puedeSaltar(puente, posicionActual + saltoActual - 1)
      saltoActual $\gets$ saltoActual - 1

    posicionActual += saltoActual
    solución = solución $\cup \{\text{posicionActual}\}$

  retornar solución



\vspace*{0.5cm}



\begin{codebox}
\Procname{$\proc{puedeSaltar}(puente, tablon)$}
    \Return $\id{tablon} \geq \proc{tamanio}(puente) \lor \proc{estaSano}(puente[tablon])$
\end{codebox}



puedeSaltar(puente, tablón)
  retornar tablón $\geq$ tamaño(puente) $\lor$ puente[tablón] está sano



\vspace*{0.5cm}



\begin{codebox}
\Procname{$\proc{esPosibleCruzar}(puente, c)$}
\li \Comment c: cantidad de tablones máximos a saltar
\li $\id{tablonesRotosConsecutivos} \gets 0$
\li \For $i \gets 0$ \To $\proc{tamanio}(puente) - 1$
\li     \Do 
            \If $\proc{puedeSaltar}(puente, i)$
\li             \Then
                    $\id{tablonesRotosConsecutivos} \gets 0$
\li         \Else
\li             $\id{tablonesRotosConsecutivos} \gets \id{tablonesRotosConsecutivos} + 1$
            \End
\li         \If $\id{tablonesRotosConsecutivos} \geq c$
\li             \Then
                    \Return $\const{false}$
            \End
        \End
\li \Return $\const{true}$   
\end{codebox}



esPosibleCruzar(puente, saltoMáximo)
  tablonesRotosConsecutivos $\gets$ 0

  para i entre 0 y tamaño(puente) - 1
    si puedeSaltar(puente, i)
      tablonesRotosConsecutivos $\gets$ 0
    sino
      tablonesRotosConsecutivos $\gets$ tablonesRotosConsecutivos + 1

    si tablonesRotosConsecutivos $\geq$ saltoMáximo
      retornar falso

  retornar verdadero


\vspace*{0.75cm} \noindent



\noindent
\textbf{1.3. Justificación de la resolución y demostración de correctitud.}

\vspace*{0.3cm}

desarrollo.


\vspace*{0.75cm} \noindent



\noindent
\textbf{1.4. Análisis de complejidad.}

\vspace*{0.3cm}

desarrollo.


\vspace*{0.75cm} \noindent



\noindent
\textbf{1.5. Partes relevantes del código (hacer referencia al apéndice).}

\vspace*{0.3cm}

desarrollo.


\vspace*{0.75cm} \noindent



\noindent
\textbf{1.6. Experimentación.}

\vspace*{0.3cm}

desarrollo.
