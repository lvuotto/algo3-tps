\subsection{Descripción del problema.}

\vspace*{0.3cm}

Este problema se trata de implementar un algoritmo que, de ser posible,
\textbf{calcule la cantidad mínima de saltos} que debe dar un participante para poder cruzar un
puente de cierto tamaño (fijo), el cual presenta \textbf{algunos de sus tablones rotos}. Estos se encuentran
marcados y \textbf{deben ser evitados}, de otro modo, el participante fracasa en su intento y puede llegar a
perder la vida. Cada uno de los participantes puede saltar una \textbf{determinada cantidad de tablones como máximo}. \medskip

El algoritmo debe decidir si dicha hazaña es posible, y de ser así,
\textbf{especificar la cantidad de saltos a dar y a qué tablones hacerlo}. \medskip

Asumimos que tanto la \textbf{longitud del puente como la cantidad de saltos máximos por participante
son valores naturales}.

\vspace*{0.5cm}

\textbf{Ejemplos:}
\begin{itemize}
  \item Para un puente de 10 tablones y un participante capaz de saltar de a 3
  tablones como máximo, teniendo los tablones 1, 4, 6, 8 y 9 rotos, la salida podría ser:
  primero saltar al tablón 3, luego al 5, luego al 7, luego al 10 y luego fuera del puente.
  \item Si en cambio, en el ejemplo anterior, el tablón 7 también estuviese roto,
  el algoritmo debe informarnos que, bajo esas condiciones, no es posible cruzar el puente.
  \item \textcolor{red}{\textbf{agregar un ejemplo mas con dibujitos y cosas lindas!}}
\end{itemize}



\subsection{Desarrollo de la idea y pseudocódigo.}

\vspace*{0.3cm}

Para resolver este problema, propusimos un algoritmo de tipo \textit{greedy}, que en cada iteración
del mismo busca siempre saltar \textbf{la mayor cantidad posible de tablones}, siempre y cuando el
tablón al que nos dirijamos esté sano. Caso contrario, se saltará al tablón anterior (ó al anterior), 
y así sucesivamente, hasta encontrar un tablón válido.
Si retrocediendo de este modo se llega a la posición en la que se encuentran el competidor, el problema
\textbf{no tiene solución}.

\vspace*{0.5cm}


\begin{codebox}
\Procname{$\proc{cruzarPuente}(n,c,puente)$}
\li \Comment n: cantidad de tablones del puente puente es el vector
\li \Comment c: cantidad de tablones máximos a saltar
\li \Comment puente: es el conjunto de tablones
\li \Comment el primer tablón es el 1
\li $\id{solucion} \gets \emptyset$
\li \If $\id{c} > \id{n}$
\li     \Then
            $\proc{agregar}(solucion, c)$
\li         \Return $\id{solucion}$
        \End
\li \If $\neg(\proc{esPosibleCruzar}(puente, c))$
\li     \Then
            \Return $\emptyset$
        \End
\li $\id{posicionActual} \gets 0$
\li \While $\id{posicionActual} \leq \id{n}$
\li     \Do
            $saltoActual \gets c$
\li         \While $\neg(\proc{puedeSaltar}(puente, posicionActual + saltoActual - 1))$
\li         \Do
                $saltoActual \gets saltoActual - 1$
            \End
\li     $\id{posicionActual} \gets \id{posicionActual} + \id{saltoActual}$
\li     $\proc{agregar}(solucion, posicionActual)$
        \End
\li \Return $\id{solucion}$
\end{codebox}


\vspace*{0.5cm}


\begin{codebox}
\Procname{$\proc{puedeSaltar}(puente, tablon)$}
    \Return $\id{tablon} \geq \proc{tamanio}(puente) \lor \proc{estaSano}(puente[tablon])$
\end{codebox}


\vspace*{0.5cm}


\begin{codebox}
\Procname{$\proc{esPosibleCruzar}(puente, c)$}
\li \Comment c: cantidad de tablones máximos a saltar
\li $\id{tablonesRotosConsecutivos} \gets 0$
\li \For $i \gets 0$ \To $\proc{tamanio}(puente) - 1$
\li     \Do
            \If $\proc{puedeSaltar}(puente, i)$
\li             \Then
                    $\id{tablonesRotosConsecutivos} \gets 0$
\li         \Else
\li             $\id{tablonesRotosConsecutivos} \gets \id{tablonesRotosConsecutivos} + 1$
            \End
\li         \If $\id{tablonesRotosConsecutivos} \geq c$
\li             \Then
                    \Return $\const{false}$
            \End
        \End
\li \Return $\const{true}$
\end{codebox}



\subsection{Justificación de la resolución y demostración de correctitud.}

\vspace*{0.3cm}

Antes de comenzar, diremos que una solución $S$ es de la forma $s \ t_1 \dots t_s$,
siendo $s$ la cantidad de saltos y $t_i$ el $i$-ésimo tablón al cual saltar. $S$ se considera
\textit{óptima} si $s$ es \textbf{mínimo}. También definiremos la relación entre soluciones $\sqsubseteq$, que,
dadas $S = s \ t_1 \dots t_s$ y $H = h \ u_1 \dots u_h$, se da del siguiente modo:
\begin{align*}
  S \sqsubseteq H \iff s \leq h \wedge \bigwedge_{i=1}^s t_i = u_i
\end{align*}

Para demostrar que el algoritmo propuesto es correcto para la resolución de este problema,
separaremos las entradas en casos y analizaremos éstos a continuación.

\begin{itemize}
  \item $c > n$: en este caso, la solución será de la forma $1 \ k$, con $k > n$. Como $c > n$, la
  solución $1 \ c$ es una solución posible (y óptima).

  \item $c \leq n$: si la instancia no tiene solución, es porque hay $c$ o más tablones
  consecutivos rotos. En tal caso, la función \textsc{esPosibleCruzar} se encarga de
  decirnos si hay solución. Más adelante demostraremos que \textsc{esPosibleCruzar}
  también es correcta. \medskip

  En cambio, si tiene solución, plantearemos el siguiente esquema:

  \begin{codebox}
  \Procname{$\proc{Greedy}(S,f,p)$}
  \li \Comment $S$ es el conjunto de tablones
  \li \Comment $f$ es una función que elige el tablón mas lejano que se puede saltar
  \li \Comment $p(A, x)$ es una función que indica si a una subsolución $A$ se le
  \li \Comment   puede agregar el tablón $x$. Es decir, si está sano y a una distancia
  \li \Comment   a lo sumo igual al salto máximo del último tablón agregado.
  \li $\id{S^{opt}} \gets \emptyset$
  \li \While $\id{S} \neq \emptyset$
  \li     \Do
    			$\id{x} \gets f(S)$
  \li  			$\id{S} \gets \id{S} \setminus \{\id{x}\}$
  \li			\If $p(S^{opt}, x)$
    				\Then
  \li					$\id{S} \gets \id{S} \setminus \{\id{c} \mid \id{c} < \id{x}\}$
  \li					$\id{S^{opt}} \gets \id{S^{opt}} \cup \{\id{x}\}$
    				\End
    		\End
  \li \Return $\id{S^{opt}}$
  \end{codebox}

  \textbf{Demostración (\textsc{cruzarPuente}):}
  \begin{itemize}
    \item Definimos como \textbf{subsolución} a un conjunto de tablones, que
    es \textbf{subconjunto de una solución óptima}.
    \item $S^{opt}$ empieza siendo $\emptyset$, que es subconjunto de cualquier
    solución óptima (si tal solución existe). Inicialmente se calcula si
    existe forma de cruzar el puente, por lo que al llegar a este punto del
    algoritmo, se puede estar seguro que existe una solución.
    \item Supongamos que estamos en la iteración k-ésima. Al iniciar
    el ciclo, $S^{opt}$ es una subsolución de la solución óptima $S^{*}$. \medskip
    
    Sea $S^{opt} \sqsubseteq S^{*}$ y $\id{x} \gets F(S)$. \medskip
    
    En esta iteración, podemos: \medskip
    
    $(a)$ No agregar $x$ a $S^{opt}$
    
    $(b)$ Agregar $x$ a $S^{opt}$ \medskip
    
    Si sucede $(a)$, entonces no se modifica $S^{opt}$, por lo tanto sigue
    siendo \textbf{subsolución de $S^{*}$}, es decir, es una \textbf{solución óptima}. \medskip
    
    Si sucede $(b)$, agregamos $x$ (aplicamos $f(S)$), es decir, el tablón más
    lejano al cuál se puede saltar. Como lo estamos agregando, vale
    $p(S^{opt}, x)$, es decir, que $x$ es un tablón sano. \medskip
    
    Sea $S'^{opt} \gets S^{opt} \cup \{x\}$. Queremos ver que existe una
    solución óptima $S'^{*}$, tal que $S'^{opt} \sqsubseteq S'^{*}$. \medskip
    
    Tenemos, nuevamente, dos casos: \medskip 
    
    $(a')$ $x \in S^{*}$
    
    $(b')$ $x \notin S^{*}$ \medskip
    
    Si sucede $(a')$, entonces $S'^{opt} \sqsubseteq S^{*}$. 
    Tomamos $S'^{*} \gets S^{*}$, una solución óptima. \medskip 
    
    Caso $(b')$. Sean $x'$ el \textbf{mínimo valor} de $S^{*} \setminus S^{opt}$, 
    $u$ el máximo valor de $S^{opt}$ (es decir, la última posición a la que
    se saltó) y $u'$ el mínimo valor de $S^{*} \setminus (S^{opt} \setminus \{x'\})$, es
    decir, el valor siguiente a $x'$ en $S^{*}$.
    
    Como $x \notin S^{*}$, $x$ no puede ser igual a $x'$. Entonces $x' > x$.
    Pero el salto de $u$ hacia $x'$ es válido, porque $S^{*}$ es una solución
    válida. Es \textbf{absurdo}, porque definimos $f$ como la función que nos
    da la mayor posición válida y en este caso retornó $x$ en lugar de $x'$.
    
    Por lo tanto, sólo puede valer $x > x'$. En este caso, consideremos a
    $S'^{*} \gets S^{*} \setminus (\{x'\} \cup \{x\})$. El salto de $u$ hacia $x$ es
    válido, porque así lo indica $p(S^{opt}, x)$ y el salto de $x$ hacia $u'$
    es menor que el de $x'$ hacia $u'$. Por lo tanto, si el salto de $x'$ es válido,
    el de $x$ también. La cantidad de saltos de $S'^{*}$ es exactamente igual a la de 
    $S^{*}$, ya que consiste simplemente en reemplazar un elemento por otro.
    
    Entonces $S^{opt} \sqsubseteq S'^{*}$, es una \textbf{subsolución de una solución óptima}.
    
    \item Al terminar el ciclo, recorremos todos los tablones y vemos que $S^{opt}$ 
    es subsolución de una solución óptima. Sea $S^{*}$ tal solución óptima.
    Si $S \gets S^{*}$ entonces $S^{opt}$ es una solución óptima.
    
    Si no, $\exists x \in (S^{*} \setminus S^{opt})$. Pero $x \in S$ y $p(S^{*}, x) \gets \const{true}$,
    por lo tanto $p(S^{opt}, x) \gets \const{true}$ y $P(X, x) \gets \const{true}$, para todo prefijo
    X de $S^{opt}$. Entonces, al momento de sacar $x$ de $S$ e intentar agregarlo a $S^{opt}$, 
    deberíamos poder, pero no es lo que sucede. 
    
    Esto no puede pasar, luego no existe $x$, y $S^{*} \gets S^{opt}$.

  \end{itemize}

\newpage

\textbf{Demostración (\textsc{esPosibleCruzar}):} 
\begin{itemize}
\item La función \textsc{esPosibleCruzar} mantiene un \textbf{contador de tablones rotos consecutivos},
inicializado en 0. Recorre todos los tablones \textbf{en orden}; si el tablón actual
está sano, resetea el contador a 0, sino (está roto) le suma 1.
Luego, si en algún momento el contador es \textbf{mayor ó igual al salto máximo posible},
devuelve \textbf{falso}. En cambio, si se recorren todos los tablones y esto no sucede, 
la función devuelve \textbf{verdadero}.

Por lo tanto, la función devuelve falso $\iff$ existe en algún momento una
cantidad de tablones rotos consecutivos mayor o igual al salto máximo. 

Veamos que esto es equivalente a que el puente pueda ser cruzado: \medskip

\textbf{Teorema:} Es posible cruzar un puente $\iff$ su cantidad de tablones rotos
consecutivos es menor al salto máximo. \medskip

\item \textbf{Demostremos la ida:}
Sea un puente de $n$ tablones, con un salto máximo de $c$ y una cantidad máxima de
tablones rotos consecutivos $k$.
Supongamos $k \geq c$.
Si es posible cruzarlo, entonces existe una solución $S$, que consiste en una
secuencia de saltos.
Sean $t_1$ el primero de los tablones y $t_c$ el tablón correspondiente a la posición $c$, entre 
los tablones consecutivos rotos (que sabemos que existe porque c $\leq$ k).
Sea $s_0 \in S$ la posición (en tierra o en un tablón) máxima anterior a $t_1$ y sea $s_1 \in S$ 
la posición inmediata siguiente a $s_0$.
Dado que $c$ es el salto máximo, entonces $s_1 - s_0 \leq c$ ó, lo que es lo
mismo, $s_0 + c \geq s_1$.
Elegimos $s_0 < t_0$, entonces $s_0 + c < t_0 + c$.
Pero esto es $s_0 + c < t_c$ y $s_1 \leq s_0 + c < t_c$.

Entonces, o bien $s_1 < t_0$ (que resulta imposible, pues se eligió $s_0$ como el máximo
menor a $t_0$), ó $t_0 \leq s_1 < t_c$, pero entonces el salto se hubiera realizado a un
tablón roto. \medskip

\textbf{¡Absurdo!}, que viene de suponer que existe una solución cuando la cantidad de
tablones rotos consecutivos es igual o mayor al salto máximo. \medskip

\item \textbf{Demostremos la vuelta:}
Sea un puente de $n$ tablones, con salto máximo $c$ y cantidad máxima de
tablones rotos consecutivos $k$, con $k < c$.
Sea $S$ la solución que consiste en caer en todos los tablones sanos y luego al
otro lado del puente (es decir, cruzarlo).

Esta solución cruza el puente completamente y para cada posición 
$s_i \in S, 1 \leq s_{i+1} - s_i \leq k$, ya que no saltea tablones sanos.
Entonces $s_{i+1} - s_i \leq k => s_{i+1} - s_i \leq c$. 

Por lo tanto todos los saltos en $S$ son válidos y la solución existe.
\end{itemize}

\end{itemize}



\subsection{Análisis de complejidad.}

\vspace*{0.3cm}

Para el análisis de complejidad nos basaremos en el pseudocódigo de la función 
\textsc{cruzarPuente}, correspondiente al ítem \textbf{3.2}.

\begin{enumerate}
  \item Todas las operaciones realizadas sobre el contenedor \verb|vector| de la \textit{STL} (size, push_back, empty y la creación de iteradores)
  toman tiempo constante $O(1)$.

  \item Las asignaciones realizadas (11, 13, 15, 16 y 17) también se realizan en tiempo constante $O(1)$.

  \item Si $c$ es mayor que $n$ (línea 6), retornamos el vector de soluciones que contiene la cantidad de saltos realizados (cero) y
  el "tablón" $c$ hacia el cual saltamos. Como dijimos anteriormente, realizar estas operaciones sobre \verb|vector|
  toma tiempo constante $O(1)$.

  \item En la línea 9, ejecutamos el condicional \verb|esPosibleCruzar(puente, c)|. La complejidad de esta función la
  analizaremos luego, pero adelantamos que es lineal, $O(n)$.

  \item En la línea 12 (tener en cuenta que en este caso vale $c \leq n$), en el peor caso ($c = 1$), realizamos $n$ iteraciones,
  pues recorremos el puente de a un tablón (siempre y cuando no haya tablones rotos, pues no tendría solución). Con $c \neq n$,
  en cualquier iteración podemos realizar a lo sumo $c$ retrocesos, pero en las siguientes esto es compensado, pues sabiendo que
  hay solución, la metodología empleada no pregunta más de 1 vez el estado de un mismo tablón. Este procedimiento tiene como peores
  casos los siguientes:
  \begin{itemize}
    \item $c = 1$, con todos los tablones sanos, de manera que haya que chequear todos los tablones del puente.

    \item $c = 2$, con un patrón de tablones sano - roto - sano, como se puede ver en la figura \textcolor{red}{\textbf{QUE DESPUES AGREGAMOS!}}

    \item $c = n$, con todos los tablones rotos salvo el primero, de manera que haya que chequear todos los tablones del puente.
  \end{itemize}

  \item La complejidad de \verb|esPosibleCruzar(puente, c)| (línea 9) es $O(n)$, pues en el peor caso recorre el puente entero.

  \item La complejidad de \verb|puedeSaltar(puente, tablón)| es $O(1)$, pues obtenemos el tamaño del puente en tiempo constante
  mediante la función \verb|size| y preguntar si un tablón está sano ó no es constante.

\end{enumerate}

Por lo tanto, la \textbf{complejidad total} del algoritmo implementado para este problema es

\begin{align*}
  O(1) + O(1) + O(n) + O(n)*O(1) = \textit{\textbf{O(n)}}
\end{align*}



\subsection{Experimentación y gráficos.}

\vspace*{0.3cm}

\subsubsection{Test 1 - con variación de tablones rotos/sanos}

En este test, fijamos el valor de $c$ (salto máximo) en 20, mientras que $n$ (cantidad de tablones) 
se inicializa en 100 y va incrementándose también de a 100, hasta alcanzar 100000. La 
distribución de los tablones rotos/sanos se genera aleatoriamente, de forma uniforme. 
Se toma el valor mínimo de cantidad de ciclos luego de 25 corridas. 

%\begin{figure}
%  \begin{center}
%    \includegraphics[scale=2]{imagenes/XXX.jpg}
%  \end{center}
%\end{figure}

\vspace*{0.5cm}


\subsubsection{Test 2 - peor caso}

En este test, fijamos el valor de $c$ (salto máximo) en 1, mientras que $n$ (cantidad de tablones) 
se inicializa en 100 y va incrementándose también de a 100, hasta alcanzar 100000. Todos los 
tablones se encuentran rotos. 
Se toma el valor mínimo de cantidad de ciclos luego de 25 corridas.

%\begin{figure}
%  \begin{center}
%    \includegraphics[scale=2]{imagenes/XXX.jpg}
%  \end{center}
%\end{figure}
\vspace*{0.5cm}


\subsubsection{Test 3 - }

En este test, fijamos el valor de $c$ (salto máximo) en 1, mientras que $n$ (cantidad de tablones) 
se inicializa en 100 y va incrementándose también de a 100, hasta alcanzar 100000. Todos los 
tablones se encuentran rotos. 
Se toma el valor mínimo de cantidad de ciclos luego de 25 corridas.

%\begin{figure}
%  \begin{center}
%    \includegraphics[scale=2]{imagenes/XXX.jpg}
%  \end{center}
%\end{figure}
