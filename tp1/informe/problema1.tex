\noindent
\textbf{1.1. Descripción del problema.}

\vspace*{0.3cm}

Este problema se trata de implementar un algoritmo que, de ser posible, calcule la
cantidad mínima de saltos que debe dar un participante para poder cruzar un puente de cierto tamaño (fijo),
el cual presenta algunos de sus tablones rotos. Estos se encuentran marcados y deben ser evitados,
de otro modo, el participante fracasa en su intento y puede llegar a perder la vida. Cada uno
de los participantes puede saltar una determinada cantidad de tablones como máximo.

El algoritmo debe decidir si dicha hazaña es posible, y de ser así, especificar
la cantidad de saltos a dar y a qué tablones hacerlo.

Asumimos que tanto la longitud del puente como la cantidad de saltos máximos por participante
son valores naturales.

\textbf{Ejemplos}
\begin{itemize}
  \item Para un puente de 10 tablones, y un participante capaz de saltar de a 3
  tablones, con los tablones 1, 4, 6, 8 y 9 rotos, la salida podría ser saltar al
  tablón 3, luego al 5, luego al 7, luego al 10 y luego afuera del puente.
  \item Si en cambio, en el ejemplo anterior, el tablón 7 también estuviese roto,
  el algoritmo debe informarnos que no es posible realizar la hazaña.
  \item agregar un ejemplo mas con dibujitos y cosas lindas
\end{itemize}


\vspace*{0.75cm} \noindent



\noindent
\textbf{1.2. Desarrollo de la idea y pseudocódigo.}

\vspace*{0.3cm}

Para resolver el problema, propusimos un algoritmo goloso, pues en cada iteración
buscamos saltar la mayor cantidad posible de tablones, siempre y cuando el tablón
al que vamos a saltar esté sano. Caso contrario, se saltará al tablón anterior.
Si todos los tablones anteriores hasta la posición actual están rotos, el problema
no tiene solución. Si el valor del salto máximo del participante es no positivo,
el problema tampoco tiene solución.

cruzarPuente(n : tablones, c : saltoMáximo)
  // n cantidad de tablones
  // c cantidad de tablones maximos a saltar
  // primer tablon = 1

  si no esPosibleCruzar(puente, saltoMáximo) entonces retornar $\emptyset$

  posicionActual $\gets$ 0
  solución $\gets \emptyset$

  mientras posicionActual $\leq$ n
    saltoActual $\gets$ saltoMáximo
    mientras no puedeSaltar(puente, posicionActual + saltoActual - 1)
      saltoActual $\gets$ saltoActual - 1

    posicionActual += saltoActual
    solución = solución $\cup \{\text{posicionActual}\}$

  retornar solución

puedeSaltar(puente, tablón)
  retornar tablón $\geq$ tamaño(puente) $\lor$ puente[tablón] está sano

esPosibleCruzar(puente, saltoMáximo)
  tablonesRotosConsecutivos $\gets$ 0

  para i entre 0 y tamaño(puente) - 1
    si puedeSaltar(puente, i)
      tablonesRotosConsecutivos $\gets$ 0
    sino
      tablonesRotosConsecutivos $\gets$ tablonesRotosConsecutivos + 1

    si tablonesRotosConsecutivos $\geq$ saltoMáximo
      retornar falso

  retornar verdadero


\vspace*{0.75cm} \noindent



\noindent
\textbf{1.3. Justificación de la resolución y demostración de correctitud.}

\vspace*{0.3cm}

desarrollo.


\vspace*{0.75cm} \noindent



\noindent
\textbf{1.4. Análisis de complejidad.}

\vspace*{0.3cm}

desarrollo.


\vspace*{0.75cm} \noindent



\noindent
\textbf{1.5. Partes relevantes del código (hacer referencia al apéndice).}

\vspace*{0.3cm}

desarrollo.


\vspace*{0.75cm} \noindent



\noindent
\textbf{1.6. Experimentación.}

\vspace*{0.3cm}

desarrollo.
