\noindent
\textbf{2.1. Descripción del problema.}

\vspace*{0.3cm}

El objetivo de este problema es, dado un conjunto de edificios, representados
por la posición de sus paredes izquierda y derecha y su altura, representar el
contorno en el horizonte formado por la superposición de los mismos, excluyendo
las líneas internas. \medskip

\textbf{Ejemplos:}
\begin{itemize}
  \item Para un conjunto de 3 edificios, cuyas coordenadas son (4, 2, 11),
  (5, 4, 7) y (9, 5, 12), dónde la primer coordenada es la ubicación de la pared
  izquierda sobre el eje X, la segunda coordenada es la altura en el eje Y y la
  tercer coordenada es la ubicación de la pared derecha sobre el eje X, la
  solución debe ser 4 2 5 4 7 2 9 5 12 0
  \item Agregar un ejemplo más e imágenes
\end{itemize}


\vspace*{0.75cm} \noindent



\noindent
\textbf{2.2. Desarrollo de la idea y pseudocódigo.}

\vspace*{0.3cm}

Para resolver este problema, primero se cargan los vértices de todos los
edificios, indicando su coordenada X, Y, aclarando si es un vértice izquierdo o
derecho.

Luego se ordenan los puntos por su coordenada X en orden ascendente y, en caso
de empate entre 2 puntos, por su coordenada Y en forma ascendente.

Se crea una lista para guardar los puntos de las soluciones y un multiconjunto
para mantener las alturas actuales de los edificios.

Se empieza cargando el punto con coordenada Y=0 y coordenada X igual al primer
punto de la lista. Y luego ese mismo punto.

Posteriormente se recorren todos los otros puntos en orden, y:
Si el punto es un vértice izquierdo: Se carga su coordenada Y en el
multiconjunto de alturas. Además si su altura es mayor a la altura del último
punto, y su coordenada X es mayor a la correspondiente del último punto, se
carga el punto generado con la coordenada X del punto actual y la coordenada Y
del punto anterior.
Si el punto es un vértice derecho: Se elimina su coordenada Y del multiconjunto
de alturas. Luego si el máximo de alturas es menor a la coordenada Y, se carga
el punto actual y luego el punto generado por la coordenada X y la altura máxima
como coordenada Y.

Finalmente se agrega el punto con coordenada Y=0 y coordenada X igual a la del
último punto cargado.


\vspace*{0.75cm} \noindent


\noindent
\textbf{2.3. Justificación de la resolución y demostración de correctitud.}

\vspace*{0.3cm}

desarrollo



\vspace*{0.75cm} \noindent



\noindent
\textbf{2.4. Análisis de complejidad.}

\vspace*{0.3cm}



desarrollo


explicar lo del multiset y las complejidades y bla.


\vspace*{0.75cm} \noindent



\noindent
\textbf{2.5. Partes relevantes del código (hacer referencia al apéndice).}

\vspace*{0.3cm}

desarrollo.


\vspace*{0.75cm} \noindent



\noindent
\textbf{2.6. Experimentación.}

\vspace*{0.3cm}

desarrollo.
