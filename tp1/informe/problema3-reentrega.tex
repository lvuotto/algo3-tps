\subsection{Informe de modificaciones}

\newpage

\subsection{Adicionales}

Se nos pide analizar la siguiente variante del problema:
\textit{¿Qué pasaría si el umbral de peligrosidad dejara de ser el mismo para todos los
camiones?}

Para resolver esta variante, en nuestro algoritmo debemos modificar las
siguientes partes:
\begin{itemize}
  \item cada camión debe tener su propio umbral de peligrosidad, por lo tanto, 
        utilizamos éste en lugar del anterior (global) en la función $\id{entra}$
  
  \item al momento de agregar un nuevo camión, deberemos seleccionar el que tenga
        el umbral más grande de los disponibles.
\end{itemize}

Esto puede afectar a la complejidad del algoritmo porque se requieren nuevos
parámetros de entrada: los camiones y su peligrosidad.
Luego, para poder obtener el camión con máximo umbral en cada iteración, es
conveniente tenerlos previamente ordenados.
Dados $k$ camiones, la nueva complejidad total sería $O(k \log(k))$ sumado a la
complejidad previa de nuestro algoritmo.
