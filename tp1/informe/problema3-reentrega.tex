\subsection{Informe de modificaciones}

\begin{itemize}
  \item se eliminaron los casos mejor y peor del algoritmo original. Ahora
  realizamos comparaciones del algoritmo base contra otras dos versiones
  que utilizan diferentes estrategias (detalladas en la sección de experimentos).

  \item se corrigió el pseudocódigo y partes del código, reflejando los
  cambios mencionados anteriormente.

  \item se realizó nuevamente el análisis de complejidad, corrigiendo
  el original que presentaba una cota errónea.
\end{itemize}


\newpage
\subsection{Adicionales}

\textit{¿Qué pasaría si el umbral de peligrosidad dejara de ser el mismo para todos los
camiones?}

\vspace*{0.25cm}

Para resolver esta variante, en nuestro algoritmo debemos modificar lo
siguiente:
\begin{itemize}
  \item cada camión debe tener su propio umbral de peligrosidad, por lo tanto,
        utilizamos éste en lugar del anterior (global) en la función $\id{entra}$

  \item al momento de agregar un nuevo camión, deberemos seleccionar el que tenga
        el umbral más grande de los disponibles.
\end{itemize}

Esto \textbf{puede afectar a la complejidad del algoritmo} porque se requieren nuevos
parámetros de entrada: los camiones y su peligrosidad.

Luego, para poder obtener el camión con máximo umbral en cada iteración, es
conveniente tenerlos previamente ordenados.

Dados $k$ camiones, la nueva complejidad total sería $O(k \log(k))$ sumado a la
complejidad previa de nuestro algoritmo.

\vspace*{0.75cm}

\textit{¿Qué cosas se podían hacer con el problema anterior para acelerar los tiempos de
ejecución que ahora ya no se pueden?}

\vspace*{0.25cm}

Dada la forma en que funciona nuestro algoritmo, \textbf{no se pueden tomar estrategias adicionales}
para acelerar los tiempos de ejecución, ni en la versión original ni en esta variante.
