\noindent
\textbf{3.1. Descripción del problema.}

\vspace*{0.3cm}

desarrollo.

\vspace*{0.75cm} \noindent

El objetivo de este problema es, distribuir N productos en C camiones, minimizando
C. Al combinarse los elementos se obtienen distintintos niveles de peligrosidad.
Para minimizar C, se debe tener en cuenta que contamos con un umbral de peligrosidad
M, el cual no debe ser superado por la suma de los niveles de peligrosidad de los
elementos que contiene.

\textbf{Ejemplos}
\begin{itemize}
  \item Dado $M = 7$ y 4 productos $p_1, p_2, p_3$ y $p_4$, con una relación
  de peligrosidad $h_{1,2} = 5, h_{1,3} = 3, h_{1,4} = 4, h_{2,3} = 6, h_{2,4} =
  3$ y $h_{3,4} = 5$, la solución óptima es utilizar 2 camiones, el primero
  transportando $p_1$ y $p_3$, y el segundo transportando $p_2$ y $p_4$.
  \item Agregar otro ejemplo.
\end{itemize}

\noindent
\textbf{3.2. Desarrollo de la idea y pseudocódigo.}

Cargamos el primer camión con el primer elemento y continuamos cargando el resto 
de los elementos hasta alcanzar el umbral M. Luego, proseguimos con la misma 
técnica en los camiones siguientes, hasta agotar los productos. De esta forma, 
obtenemos una solución candidata (no necesariamente óptima).

En las siguientes iteraciones, si la cantidad de camiones utilizados hasta el 
momento es C-1 y al agregar el pŕoximo elemento pasamos a utilizar C camiones, 
esa solución queda descartada.


biohazard(n, l, m)
  while noVacio(conjProductos(p))
    camion_actual <- conjVacio(Nat)
    lista(camion) lista_camiones <- listaVacia()
    while (sumatoria_coeficientes(camion_actual) <= m)
      agregar(camion_actual, dameUno(conjProductos(p))
      producto q <- dameUno(conjProductos(p))
      conjProductos(p) - dameUno(conjProductos(p))
      if (sumatoria_coeficientes(camion_actual) > m)
        camion_actual - dameUno(conjProductos)
        agregar(conjProductos(p), q)
    agregarAtras(l, camion_actual)
  return 
  
  
  El = conj(e)  {e = elementos}
  P = conj(p)   {p = peligrosidad de 2 elementos}
  
  biohazard (Elementos, LimitePeligrosidad, CoefPeligrosidades)

  SolOptima = Elementos.cantidad
  Camiones
    Foreach( e in Elementos){
      Camion c
      c.agregar(e)
      TempEl = Elementos
      TempEl.eliminar(e)
      Foreach(i in iteracionesde(TempEl)){
        If(IndiceP(c.agregar(i)) >= LimitePeligrosidad){
          Camiones.agregar(Camion c2))
          if(Camiones.cantidad >= SolOptima){
            Next
          }
        }
        Else{
          c.agregar(i)
        }
      }
    }
    
  

 
\vspace*{0.3cm}

desarrollo.


\vspace*{0.75cm} \noindent


\noindent
\textbf{3.3. Justificación de la resolución y demostración de correctitud.}

\vspace*{0.3cm}

desarrollo.


\vspace*{0.75cm} \noindent



\noindent
\textbf{3.4. Análisis de complejidad.}

\vspace*{0.3cm}

desarrollo.


\vspace*{0.75cm} \noindent



\noindent
\textbf{3.5. Partes relevantes del código (hacer referencia al apéndice).}

\vspace*{0.3cm}

desarrollo.


\vspace*{0.75cm} \noindent



\noindent
\textbf{3.6. Experimentación.}

\vspace*{0.3cm}

desarrollo.
