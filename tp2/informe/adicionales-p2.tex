Utilizando alfiles en lugar de caballos, sucede lo siguiente: 

\begin{itemize}
        \item Cambian los "vecinos" de un casillero. Ahora, dos casilleros (o nodos) del tablero
        serán vecinos si forman parte de la misma diagonal, pues en 1 sólo movimiento, un alfil
        puede acceder a cualquier casillero de la misma, volviéndolos adyacentes.

        \item Podemos saber más rápidamente si el problema tiene solución o no (en $O(k)$, 
        con $k$ = cantidad de alfiles). Si existe al menos un alfil ubicado en un casillero de color   
        distinto al resto, el problema no tiene solución, pues, dado los movimientos que realiza un 
        alfil, no puede pasar de un casillero negro a uno blanco o viceversa. En nuestro caso, para 
        chequear el "color" de un casillero, basta con ver si las coordenadas $i, j$ de los alfiles 
        coinciden en paridad.

        \item Un alfil necesita tan sólo 2 movimientos para acceder a cualquier casillero del 
        tablero (siempre y cuando sea un casillero válido, es decir, que coincida en color con 
        el casillero inicial), por lo tanto, en $O(1)$ podemos saber cuánto le cuesta a un
        alfil moverse a determinado casillero, siendo $O(k)$ el costo de averiguarlo para todos los    
        alfiles. Sin embargo, la complejidad total del algoritmo va a seguir dependiendo del tamaño del 
        tablero ($n^2$), pues debemos chequear todos para saber en cúal minimizamos la cantidad de       
        movimientos totales. Luego, la complejidad seguirá siendo $O(k*n^2)$.
\end{itemize}
