\subsection{Descripción del problema.}

\vspace*{0.3cm}

En este problema, debemos encontrar una \textbf{secuencia de vuelos que
permita, dada una ciudad de origen, otra de destino y un listado de vuelos, 
viajar del origen al destino, llegando lo más temprano posible}.

Esta secuencia \textbf{debe comenzar con un vuelo que parta de la ciudad 
origen}. Luego, \textbf{cada vuelo debe salir de la ciudad de arribo del 
anterior, dejando libre un lapso de, al menos, 2 horas}, para finalmente 
llegar a la ciudad destino. 

De todas las posibles combinaciones válidas de vuelos, \textbf{la escogida 
debe ser la que llegue antes al destino}.

También se deberá informar cuando el recorrido no puede realizarse.

\vspace*{0.5cm}

\textbf{Ejemplos:}
%\begin{itemize}
  \textcolor{red}{\textbf{completar!}}
%\end{itemize}



\newpage
\subsection{Desarrollo de la idea y pseudocódigo.}

\vspace*{0.3cm}

Para resolverlo, se propone \textbf{ordenar todos los vuelos disponibles 
en base al horario de llegada, en forma ascendente}. Hecho esto, se recorre 
cada vuelo y se pregunta si es posible

\begin{enumerate}
  \item llegar a la ciudad de origen \textbf{2 horas antes del horario de partida}.
  \item que la ciudad de partida de dicho vuelo es la ciudad de origen del cliente.
\end{enumerate}

De valer alguno de estos ítems, \textbf{si no hay otro vuelo que llegue antes 
a su destino}, se procede a marcar dicho vuelo como el primero en llegar y a 
su predecesor como el primero en llegar a la ciudad de origen. Caso contrario, 
el vuelo es ignorado.

Una vez recorridos todos, si existe algún vuelo que llegue al destino, existe 
solución, la cual se construye a partir de recorrer los vuelos predecesores.
\textbf{Dicha cadena tiene como primer vuelo, aquel que parte desde la ciudad 
de origen del cliente.}


\begin{codebox}
\Procname{$\proc{planDeVuelo}(vuelos,origen,destino)$}
\li \Comment $\id{vuelos}$ es un arreglo de tipo Vuelo, donde Vuelo es una 
\li \Comment estructura conformada por una ciudad origen, una ciudad destino, 
\li \Comment un horario de partida y otro de arribo.
\li
\li \Comment $\id{rutas}$ es un diccionario de ciudades en vuelos, en el cual, 
\li \Comment para cada ciudad, nos dice cual es el vuelo que llega antes.
\li $\proc{ordenar}(vuelos)$
\li $\id{rutas} \gets \emptyset$
\li \While $\neg (\proc{vacio}(vuelos))$ 
\li     \Do
            $\id{vuelo} \gets \proc{primero}(vuelos)$
\li         \If $\neg (\proc{existe}(\proc{destino}(vuelo),rutas)) \land
                \proc{puedeTomar}(vuelo,rutas,origen)$ 
\li             \Then
                    $\id{rutas[\proc{destino}(vuelo)]} \gets vuelo$
                \End
\li         $\id{vuelos} \gets \id{vuelos} \setminus \{vuelo\}$
        \End
\li \If $\proc{existe}(destino,rutas)$ 
\li     \Then
            \Return $\proc{armarPila}(rutas,destino)$
\li     \Else
\li         \Return $\emptyset$
        \End
\end{codebox}


\vspace*{0.25cm}


\begin{codebox}
\Procname{$\proc{puedeTomar}(vuelo,rutas,origen)$}
\li \Return $\proc{origen}(vuelo) \isequal \id{origen} \lor
            (\proc{existe}(\proc{origen}(vuelo),rutas) \land
             \proc{llegada}(rutas[\proc{origen}(vuelo)]) \leq
             \id{u} - 2)$
\end{codebox}


\vspace*{0.25cm}


\begin{codebox}
\Procname{$\proc{armarPila}(rutas,destino)$}
\li $\id{lista} \gets \emptyset$
\li \If $\proc{existe}(destino,rutas)$
\li     \Then
            $\id{vuelo} \gets \id{rutas[destino]}$
\li         $\proc{agregarAdelante}(vuelo,lista)$
\li         \While $\exists \hspace{0.07cm} \id{vuelo.predecesor}$ 
\li             \Do
                   $\id{vuelo} \gets \id{vuelo.predecesor}$  
\li                $\proc{agregarAdelante}(vuelo,lista)$  
                \End
\li     \Else
\li         \Return $\id{lista}$
        \End
\end{codebox}



\newpage
\subsection{Justificación de la resolución y demostración de correctitud.}

\vspace*{0.3cm}

\textcolor{red}{\textbf{completar!}}



\newpage
\subsection{Análisis de complejidad.}

\vspace*{0.3cm}

\textcolor{red}{\textbf{completar!}}



\newpage
\subsection{Experimentación y gráficos.}

\vspace*{0.3cm}

\subsubsection{Test 1 - benchmark caso aleatorio}

\textcolor{red}{\textbf{completar!}}


\newpage
\subsubsection{Test 2 - benchmark del peor caso}

\textcolor{red}{\textbf{completar!}}


\newpage
\subsubsection{Test 3 - benchmark del mejor caso}

\textcolor{red}{\textbf{completar!}}
