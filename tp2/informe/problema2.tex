\subsection{Descripción del problema.}

\vspace*{0.3cm}

\textcolor{red}{\textbf{completar!}}

\vspace*{0.5cm}

\textbf{Ejemplos:}
%\begin{itemize}
  \textcolor{red}{\textbf{completar!}}
%\end{itemize}



\newpage
\subsection{Desarrollo de la idea y pseudocódigo.}

\vspace*{0.3cm}


Para resolver el problema se creará un tablero de $n$x$n$ casilleros por cada
caballo.
En cada tablero se calculará cuanto le cuesta a dicho caballo llegar a cada casillero,
aplicando BFS desde el casillero donde comienza, quedando inválidos los casilleros
a los cuales no se pueden alcanzar.
Luego se recorren todos los casilleros sumando el valor de todos los tableros, si
todos son alcanzables, obteniendo así el costo de cada casillero para todos los caballos.
De existir, el mínimo de estos valores será el casillero al cuál pueden alcanzar todos 
los caballos en menor cantidad de saltos.

\begin{codebox}
\Procname{$\proc{puntoDeEncuentro}(caballos, n)$}
\li $\id{tableros} \gets \emptyset$
\li \For $caballo \in caballos$
\li   \Do
\li     $\proc{agregar}(\proc{crearTablero}(caballo, \id{n}), tableros)$
      \End
\li $\id{i} \gets 0$
\li $\id{j} \gets 0$
    min_i = -1
    min_j = -1
    min = -1
\li \While $\id{i} < \id{n}$
\li   \Do
\li     \While $\id{j} < \id{n}$
\li       \Do
\li         $\id{sum} \gets 0$
\li         $\id{caballo} \gets 0$
\li         \While $\id{caballo} < \proc{tamaño}(caballos)$
\li           \Do
                if (tableros[caballo][i][j] == -1)
                  sum = -1
                else if (sum != -1)
                  sum += tableros[caballo][i][j]
                end
                caballo++
              \End
            if (sum != -1 && (min == -1 || sum < min))
              min = sum
              min_i = i
              min_j = j
            end
            j++
        \End
        i++
      \End
    if min == -1
      \Return 'no'
    else
\li \Return min min_i min_j
\end{codebox}


crearTablero

\newpage
\subsection{Justificación de la resolución y demostración de correctitud.}

\vspace*{0.3cm}

\textcolor{red}{\textbf{completar!}}



\newpage
\subsection{Análisis de complejidad.}

\vspace*{0.3cm}

\textcolor{red}{\textbf{completar!}}



\newpage
\subsection{Experimentación y gráficos.}

\vspace*{0.3cm}

\subsubsection{Test 1 - benchmark caso aleatorio}

\textcolor{red}{\textbf{completar!}}


\newpage
\subsubsection{Test 2 - benchmark del peor caso}

\textcolor{red}{\textbf{completar!}}


\newpage
\subsubsection{Test 3 - benchmark del mejor caso}

\textcolor{red}{\textbf{completar!}}
