\subsection{Descripción del problema.}

\vspace*{0.3cm}

\textcolor{red}{\textbf{completar!}}

\vspace*{0.5cm}

\textbf{Ejemplos:}
%\begin{itemize}
  \textcolor{red}{\textbf{completar!}}
%\end{itemize}



\newpage
\subsection{Desarrollo de la idea y pseudocódigo.}

\vspace*{0.3cm}

\textcolor{red}{\textbf{completar!}}

En este problema vamos a considerar la red de computadoras como un grafo, donde
las computadoras serán los vértices, las conexiones las aristas y el peso de cada
arista será el costo de hacer dicha conexión.

Comenzaremos por corroborar que exista una solución. Para esto el grafo deberá 
ser conexo y tener por lo menos tantas aristas como vértices. Si no fuese conexo
no se podrían unir todas las computadoras, y si tiene menos aristas que vértices,
el grafo sería un arbol y nunca se podría formar un anillo.

Suponiendo que tiene solución, se procederá a calcular el árbol generador mínimo
mediante el algoritmo de Prim, y para formar el anillo se formará un ciclo agergando
la arista de menor peso no incluida en árbol generador mínimo.

\newpage
\subsection{Justificación de la resolución y demostración de correctitud.}

\vspace*{0.3cm}

\textcolor{red}{\textbf{completar!}}

\begin{codebox}
\Procname{$\proc{anillar}(grafo)$}
\li \If $\proc{noTieneSolucion}(grafo)$
\li     \Then
          \Return $\const{false}$
        \End
\li  \Return $\proc{completarAnillo}(\proc{Prim}(grafo))$   
\end{codebox}

\begin{codebox}
\Procname{$\proc{noTieneSolucion}(grafo)$}
\li \Return $\proc{esConexo}(grafo) \land 
\proc{tamaño}(\proc{nodos}(grafo)) \leq \proc{tamaño}(\proc{vertices}(grafo))$
\end{codebox}

\begin{codebox}
\Procname{$\proc{Prim}(grafo)$}

AGREGAME EL PSEUDO CODIGO DE PRIM DE LA TEORICA VITEH!!!!

\end{codebox}

\begin{codebox}
\Procname{$\proc{esConexo}(grafo)$}

HACER BFS VITEH!

\end{codebox}

\begin{codebox}
\Procname{$\proc{completarAnillo}(agm, grafo)$}
\li $\id{X} \gets \proc{vértices}(grafo) \setminus \proc{vértices}(agm)$
\li $\id{m} \gets \proc{dame_uno}(X)$
\li $\id{X} \gets \id{X} \setminus \{m\}$
\li \While $\neg \proc{vacío}(X)$ \Do
\li   $\id{x} \gets \proc{dame_uno}(X)$
\li   \If $\proc{peso(x)} < \proc{peso(m)}$ \Then
\li     $\id{m} \gets \id{x}$
      \End
\li   $\id{X} \gets \id{X} \setminus \{x\}$
    \End
\li \Return $\proc{agregar}(agm, m)$
\end{codebox}

\newpage
\subsection{Análisis de complejidad.}

\vspace*{0.3cm}

\textcolor{red}{\textbf{completar!}}



\newpage
\subsection{Experimentación y gráficos.}

\vspace*{0.3cm}

\subsubsection{Test 1 - benchmark caso aleatorio}

\textcolor{red}{\textbf{completar!}}


\newpage
\subsubsection{Test 2 - benchmark del peor caso}

\textcolor{red}{\textbf{completar!}}


\newpage
\subsubsection{Test 3 - benchmark del mejor caso}

\textcolor{red}{\textbf{completar!}}
