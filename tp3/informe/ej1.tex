\subsection{Relación entre k-PMP y el problema 3 del TP 1.}
\vspace*{0.3cm}

Una relación posible entre k-PMP y el problema 3 del TP1 podría ser que, si
k-PMP tiene una solución válida, donde el peso total sea mínimo y además el peso total de la $k$-partición sea menor o igual al umbral $M$ de peligrosidad del problema 3 del TP1, esto significa que el problema 3 del TP1 tiene solución con $k$ camiones, aunque no necesariamente sea la solución óptima a este problema.

Es decir, quizás el problema 3 del TP1 pueda resolverse utilizando menos de $k$
camiones, pero no podemos inferir esta información a partir del resultado del
k-PMP, sólo saber si funciona para $k$.


\newpage
\subsection{Relación entre k-PMP y el problema de coloreo de los vértices de
            un grafo.}
\vspace*{0.3cm}

Sea $G = (V, E)$ un grafo simple. Dado un entero $k$ y una función de peso $w$
que, dado un eje $e \in E$ le asigna a $e$ un peso \textbf{estrictamente positivo},
la relación entre el problema de $k-PMP$ y el problema de coloreo de los vértices
de un grafo puede resumirse en la ecuación siguiente:

\begin{center}
  k-PMP($G$) = $0 \iff G$ es $k$-coloreable
\end{center}

Esto vale pues, si la $k$-partición de $G$ de \textbf{peso mínimo} hallada
tiene peso $0$, significa que cada conjunto de la partición (subconjuntos de $V$)
no posee aristas \textit{intrapartición}, es decir, sus extremos no se encuentran en
un mismo conjunto, por lo tanto el peso de cada uno de estos conjuntos será
\textbf{nulo} y en consecuencia también lo será la suma total de los pesos de
las aristas \textit{intrapartición}. Si esto sucede, entonces el grafo $G$ es
$k$-coloreable, pues esto significa que $V$ puede dividirse en, a lo sumo, $k$
conjuntos, siendo $k$ la cantidad máxima de colores a utilizar, de manera que no
haya vértices adyacentes en estos conjuntos, pues el \textit{coloreo} no nos permite
utilizar colores iguales para vértices adyacentes.

De la misma forma, si el grafo $G$ es $k$-coloreable, significa que utilizando
como máximo $k$ colores podemos $colorear$ los vértices del mismo, esto implica
que $V \in G$ puede dividirse en $i$ conjuntos, con $i \in {1, \dots, k}$, siendo $i$
la cantidad de colores utilizada para colorear los vértices del grafo, de manera
que en cada uno de estos conjuntos no haya vértices adyacentes, pues los agrupamos
según el color que le fue asignado a cada vértice. Esto implica que el peso de
las aristas \textit{intrapartición} es nulo para todos los $i$ conjuntos, pues éstas no
existen, ya que las aritas presentes en $G$ no conectan vértices pertenecientes a
un mismo subconjunto de $V$. Por lo tanto, al ser el peso del conjunto $i$ nulo,
la suma de los pesos de las aristas \textit{intrapartición} será también nulo, por lo
que el peso de la $k$-partición de $G$ será nulo, por lo tanto vale que $k-PMP(G) = 0$.

Es importante destacar que la relación planteada anteriormente y expresada mediante
la ecuación descripta más arriba sólo vale si la función de peso $w$ asigna a las
aristas valores positivos, caso contrario la partición peso mínimo podría ser una
cuyo peso sea negativo, la cantidad de aristas \textit{intrapartición} podría variar y el
\textit{coloreo} del grafo $G$ necesitaría más de $k$ colores para llevarse a cabo.



\newpage
\subsection{Situaciones de la vida real que pueden modelarse con k-PMP.}
\vspace*{0.3cm}

La ciudad de \textit{Malos Ayres} se encuentra dividida en $k$ barrios, para una mejor administración. Una de las compañías encargadas de brindar servicios de telecomunicaciones (internet, telefonía, etc.), es \textit{Bad Conexion S.A.}.
La compañia debe instalar $n$ antenas en la ciudad, de manera que pueda cubrir las necesidades de conexión de la misma.

En la ciudad existe también un Ente Regulador de Telecomunicaciones (\textit{ENT}), que sanciona a las compañías cuyo servicio prestado posea una calidad por debajo del mínimo establecido por ley. Dicha calidad de servicio se mide en base al \textit{grado de conectividad global} que otorga cada compañia, el cual se calcula sumando el \textit{grado de conectividad local} asociado a cada uno de los $k$ barrios de la ciudad. \textit{Bad Conexion S.A.} encontró un hueco legal en esta disposición y decidió aprovecharse del mismo, aplicando un \textit{plan de ahorro en infraestructura}, aunque ésto tenga traiga aparejado una muy probable caída en la calidad del servicio en algunas localidades.

Para esto, decide distribuir $n$ antenas entre los $k$ barrios, de manera tal que el \textit{grado de conectividad global} alcanzado sea el mínimo que esté por encima del fijado, evitando así ser sancionada por el \textit{ENT}, pero además reduciendo costos, ya que este grado puede ser mantenido aún brindando un servicio deficiente en algunos de los sectores de la ciudad.

Vemos que es posible resolver este escenario ficticio que podría llegar a plantearse en la vida real utilizando k-PMP, si planteamos el grafo de la siguiente manera:

\begin{itemize}
\item Los $n$ nodos del grafo son las antenas a distribuir en las $k$ regiones.

\item $k$ es la cantidad de regiones/barrios en las que se divide la ciudad.

\item Dado un par de nodos $i, j$, la relación entre los mismos es simétrica, dado que lo que calcula la \textit{función de ``peso''} es el \textit{nivel de conectividad} y se aplica a la arista incidente a los mismos.
\end{itemize}

De esta forma, para calcular el ``peso'' (o nivel de conectividad) de una $k$-partición, se suman los ``pesos'' de las aristas intrapartición, obteniendo así el \textit{grado de conectividad global}. Si este grado \textit{global} se mantiene por encima del mínimo requerido, es una solución válida \textit{Bad Conexion S.A.}, pudiendo así lograr su cometido.
