\subsection{Relación entre k-PMP y el problema 3 del TP 1.}
\vspace*{0.3cm}

Una relación posible entre k-PMP y el problema 3 del TP1 podría ser que, si
k-PMP tiene una solución, de manera tal que el peso total de la $k$-partición sea menor o igual al umbral M de peligrosidad del P3 del TP1, esto significa que el P3 del TP1 tiene solución con k camiones, aunque no necesariamente sea la solución óptima a este problema.

Es decir, quizás el P3 del TP1 pueda resolverse utilizando menos de $k$ camiones, pero no podemos inferir esta información a partir del resultado del k-PMP, sólo saber si funciona para k.


\newpage
\subsection{Relación entre k-PMP y el problema de coloreo de los vértices de
            un grafo.}
\vspace*{0.3cm}

Sea $G = (V, E)$ un grafo simple. Dado un entero $k$ y una función de peso $w$
que, dado un eje $e \in E$ le asigna a $e$ un peso \textbf{estrictamente positivo},
la relación entre el problema de $k-PMP$ y el problema de coloreo de los vértices
de un grafo puede resumirse en la ecuación siguiente:

\begin{center}
  k-PMP($G$) = $0 \iff G$ es $k$-coloreable
\end{center}

Esto vale pues, si la $k$-partición de $G$ de \textbf{peso mínimo} hallada
tiene peso $0$, significa que cada conjunto de la partición (subconjuntos de $V$)
no posee aristas \textit{intrapartición}, es decir, sus extremos no se encuentran en
un mismo conjunto, por lo tanto el peso de cada uno de estos conjuntos será
\textbf{nulo} y en consecuencia también lo será la suma total de los pesos de
las aristas \textit{intrapartición}. Si esto sucede, entonces el grafo $G$ es
$k$-coloreable, pues esto significa que $V$ puede dividirse en, a lo sumo, $k$
conjuntos, siendo $k$ la cantidad máxima de colores a utilizar, de manera que no
haya vértices adyacentes en estos conjuntos, pues el \textit{coloreo} no nos permite
utilizar colores iguales para vértices adyacentes.

De la misma forma, si el grafo $G$ es $k$-coloreable, significa que utilizando
como máximo $k$ colores podemos $colorear$ los vértices del mismo, esto implica
que $V \in G$ puede dividirse en $i$ conjuntos, con $i \in {1, \dots, k}$, siendo $i$
la cantidad de colores utilizada para colorear los vértices del grafo, de manera
que en cada uno de estos conjuntos no haya vértices adyacentes, pues los agrupamos
según el color que le fue asignado a cada vértice. Esto implica que el peso de
las aristas \textit{intrapartición} es nulo para todos los $i$ conjuntos, pues éstas no
existen, ya que las aritas presentes en $G$ no conectan vértices pertenecientes a
un mismo subconjunto de $V$. Por lo tanto, al ser el peso del conjunto $i$ nulo,
la suma de los pesos de las aristas \textit{intrapartición} será también nulo, por lo
que el peso de la $k$-partición de $G$ será nulo, por lo tanto vale que $k-PMP(G) = 0$.

Es importante destacar que la relación planteada anteriormente y expresada mediante
la ecuación descripta más arriba sólo vale si la función de peso $w$ asigna a las
aristas valores positivos, caso contrario la partición peso mínimo podría ser una
cuyo peso sea negativo, la cantidad de aristas \textit{intrapartición} podría variar y el
\textit{coloreo} del grafo $G$ necesitaría más de $k$ colores para llevarse a cabo.



\newpage
\subsection{Situaciones de la vida real que pueden modelarse con k-PMP.}
\vspace*{0.3cm}
\textcolor{red}{\textbf{completar!}}

