\subsection{Descripción del algoritmo implementado.}
\vspace*{0.3cm}
\textcolor{red}{\textbf{completar!}}

Planteamos las siguientes vecindades:
- "mover": Una partición es vecina de otra si consiste en mover un único vértice de un conjunto a otro.
- "intercambiar": Una partición es vecina de otra si consiste en intercamiar 2 vértices de distintos conjuntos.

Los algoritmos consisten en empezar con una solución y probar cuál de todas las vecinas tiene un peso menor, y continuar desde esa solución. Cuando no se pueda mejorar más, se termina.

Pseudo mover:
kpmp_mover(particion) {
	pesoMin = peso(particion)
	sePuedeMejorar = true

	mientras sePuedeMejorar {
		por cada vértice en vértices(particion) {
			conjuntoDelVertice = buscarConjunto(particion, vertice)
			pesoSinVertice = peso(particion) - costo(conjuntoDelVertice, vertice)

			por cada conjunto en conjuntos(particion) excepto conjuntoDelVertice {
				peso = pesoSinVertice + costo(conjunto, vertice)

				si peso < pesoMin {
					pesoMin = peso
					verticeMin = vertice
					conjuntoViejo = conjuntoDelVertice
					conjuntoNuevo = conjunto
				}
			}
		}

		si pesoMin < peso(particion) {
			sacarVertice(conjuntoViejo, verticeMin)
			agregarVertice(conjuntoNuevo, verticeMin)
		} sino {
			sePuedeMejorar = false
		}
	}
}

Pesudo intercambiar:
kpmp_switch(particion) {
	pesoMin = peso(particion)
	sePuedeMejorar = true

	mientras sePuedeMejorar {
		por cada vértice1 en vértices(particion) {
			conjuntoDelVertice1 = buscarConjunto(particion, vertice)

			por cada vértice2 en vértices(partición) excepto que compartan conjunto {
				conjuntoDelVertice2 = buscarConjunto(particion, vertice)

				peso = peso(particion)
					- costo(conjuntoDelVertice1, vertice1)
					- costo(conjuntoDelVertice2, vertice2)
					+ costo(conjuntoDelVertice1, vertice2)
					+ costo(conjuntoDelVertice2, vertice1)


				si peso < pesoMin {
					pesoMin = peso
					verticeMin1 = vertice1
					verticeMin2 = vertice2
					conjuntoMin1 = conjuntoDelVertice1
					conjuntoMin2 = conjuntoDelVertice2
				}
			}
		}

		si pesoMin < peso(particion) {
			sacarVertice(conjuntoDelVertice1, verticeMin1)
			agregarVertice(conjuntoDelVertice1, verticeMin2)
			sacarVertice(conjuntoDelVertice2, verticeMin2)
			agregarVertice(conjuntoDelVertice2, verticeMin1)
		} sino {
			sePuedeMejorar = false
		}
	}
}

\newpage
\subsection{Análisis de complejidad del peor caso de una iteración del
            algoritmo de búsqueda local.}
\vspace*{0.3cm}
\textcolor{red}{\textbf{completar!}}

n = |vertices|
m = |aristas|
k = |conjuntos|

El algoritmo mover en cada iteración recorre todos los vértices, por cada uno recorre todos los conjuntos, y por cada uno calcula un costo, que es recorrer los vértices. Igualmente como en el goloso, se terminan haciendo n comparaciones en total ya que los conjuntos son disjuntos.
Al final, si hay uno mejor, se mueve el vértice generando 2 cálculos de costos.
En total: O((n * (k + n)) + n) = O(n^2 + kn)

EL algoritmo intercambiar, recorre todos los vértices y, por cada uno, nuevamente los otros vértices, dentro de estos, calcula los costos.
Al final, si hay uno mejor, se intercambian los vértices generando 4 cálculos de costos.
Total: O(n * n * n + n) = O(n^3)

\newpage
\subsection{Experimentación y gráficos.}
\vspace*{0.3cm}
\textcolor{red}{\textbf{completar!}}
