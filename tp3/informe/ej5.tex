\subsection{Descripción del algoritmo implementado.}
\vspace*{0.3cm}

El algoritmo de \textbf{GRASP} consiste en utilizar una
\textit{heurística golosa aleatorizada} para obtener una primera solución
y luego ir mejorándola aplicándole una \textit{heurística de búsqueda local}.

Esto se repite hasta llegar a un \textit{criterio de terminación} dado,
quedándose con la mejor solución encontrada hasta el momento.

El algoritmo que diseñamos utiliza el algoritmo goloso planteado en el
ejercicio 3 pero, para la selección de candidatos en la \textit{heurística golosa aleatorizada}, se plantean 2 alternativas:
\begin{enumerate}
\item En lugar de elegir la arista más pesada, se elige una al azar de las $X$ más pesadas.

\item En lugar de poner el vértice en el conjunto que genere el menor peso, se elige un conjunto al azar de los $X$ que generan menor peso.
\end{enumerate}

Luego se mejora con alguno de los algoritmos de búsqueda local planteados
en el ejercicio 4.

Y, finalmente, se diseñaron 2 criterios de terminación:
\begin{enumerate}
\item Elegir la mejor solución luego de $X$ iteraciones.

\item Elegir la mejor solución luego de que éste se encuentre como solución $X$ veces. Si se encuentra una solución mejor, se resetea el contador.
\end{enumerate}


Se probó con cada estrategia de la \textit{heurística golosa aleatorizada} por separado y luego con ambas.

Todos estos resultados fueron probados con ambos algoritmos de localidad.

Por último también se combinaron todos con los 2 criterios de terminación, con distintos valores para las variables $X$ (independientes) detalladas anteriormente.

\vspace*{0.35cm}

\textbf{Pseudocódigo del algoritmo GRASP:}

\vspace*{0.1cm}

\begin{verbatim}
kpmp_grasp(funcionGoloso, funcionLocal, grafo, cantidadDeConjuntos, criterioDeTerminacion, X) {
    particion = funcionGoloso(grafo, cantidadDeConjuntos)
    funcionLocal(particion)
    pesoMin = peso(particion)
    si criterioDeTerminacion = MEJOR_LUEGO_DE_X_VECES {
        repetir X veces {
            nuevaParticion = funcionGoloso(grafo, cantidadDeConjuntos)
            funcionLocal(particion)
            si peso(nuevaParticion) < pesoMin {
                pesoMin = peso(nuevaParticion)
                particion = nuevaParticion
            }
        }
    } sino criterioDeTerminacion = CUANDO_EL_MEJOR_SE_REPITA_X_VECES {
        nuevaParticion = funcionGoloso(grafo, cantidadDeConjuntos)
        funcionLocal(nuevaParticion)
        contadorRepeticiones = 1
        si peso(nuevaParticion) < pesoMin {
            pesoMin = peso(nuevaParticion)
            particion = nuevaParticion
            contadorRepeticiones = 1
        } sino peso(nuevaParticion = pesoMin) {
            contadorRepeticiones += 1
        }
    }
    return particion
}
\end{verbatim}



\newpage
\subsection{Experimentación y gráficos.}
\vspace*{0.3cm}
\textcolor{red}{\textbf{completar!}}
