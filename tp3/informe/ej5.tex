\subsection{Descripción del algoritmo implementado.}
\vspace*{0.3cm}
\textcolor{red}{\textbf{completar!}}

El algoritmo GRASP que diseñamos tiene 2 criterios de terminación:
- Elegir el mejor luego de X iteraciones.
- Elegir el mejor luego de que éste se encuentre como solución X veces, si se encuentra una solución mejor, se resetea el contador.

Para la selección de candidatos, el algoritmo goloso se plantean 2 formas de aleatorizar:
- En lugar de elegir la arista más pesada, se elige una al azar de las X más pesadas.
- En lugar de poner el vértice en el conjunto que genere el menor peso, se elige un conjunto al azar de los X que generan menor peso.

Se probó con cada una por separado y con las dos juntas.

También se probaron con ambos algoritmos de localidad.

Pseudo:
kpmp_grasp(funcionGoloso, funcionLocal, grafo, cantidadDeConjuntos, criterioDeTerminacion, X) {
	particion = funcionGoloso(grafo, cantidadDeConjuntos)
	funcionLocal(particion)
	pesoMin = peso(particion)

	si criterioDeTerminacion = MEJOR_LUEGO_DE_X_VECES {
		repetir X veces {
			nuevaParticion = funcionGoloso(grafo, cantidadDeConjuntos)
			funcionLocal(particion)

			si peso(nuevaParticion) < pesoMin {
				pesoMin = peso(nuevaParticion)
				particion = nuevaParticion
			}
		}
	} sino criterioDeTerminacion = CUANDO_EL_MEJOR_SE_REPITA_X_VECES {
		nuevaParticion = funcionGoloso(grafo, cantidadDeConjuntos)
		funcionLocal(nuevaParticion)
		contadorRepeticiones = 1

		si peso(nuevaParticion) < pesoMin {
			pesoMin = peso(nuevaParticion)
			particion = nuevaParticion
			contadorRepeticiones = 1
		} sino peso(nuevaParticion = pesoMin) {
			contadorRepeticiones += 1
		}
	}

	return particion
}

\newpage
\subsection{Experimentación y gráficos.}
\vspace*{0.3cm}
\textcolor{red}{\textbf{completar!}}
