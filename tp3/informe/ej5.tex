\subsection{Descripción del algoritmo implementado.}
\vspace*{0.3cm}
\textcolor{red}{\textbf{completar!}}

El algoritmo de \textbf{GRASP} diseñado tiene 2 criterios de terminación:
\begin{enumerate}
\item Elegir la mejor solución luego de $X$ iteraciones.

\item Elegir la mejor solución luego de que éste se encuentre como solución $X$ veces. Si se encuentra una solución mejor, se resetea el contador.
\end{enumerate}

Para la selección de candidatos, en la \textit{heurística golosa aleatorizada} se plantean 2 formas:
\begin{enumerate}
\item En lugar de elegir la arista más pesada, se elige una al azar de las $X$ más pesadas.

\item En lugar de poner el vértice en el conjunto que genere el menor peso, se elige un conjunto al azar de los $X$ que generan menor peso.
\end{enumerate}

Se probó con cada estrategia una por separado y luego con ambas.

También se probaron con ambos algoritmos de localidad.

\vspace*{0.5cm}

\textbf{Pseudocódigo:}

\vspace*{0.3cm}

\begin{verbatim}
kpmp_grasp(funcionGoloso, funcionLocal, grafo, cantidadDeConjuntos, criterioDeTerminacion, X) {
    particion = funcionGoloso(grafo, cantidadDeConjuntos)
    funcionLocal(particion)
    pesoMin = peso(particion)

    si criterioDeTerminacion = MEJOR_LUEGO_DE_X_VECES {
        repetir X veces {
            nuevaParticion = funcionGoloso(grafo, cantidadDeConjuntos)
            funcionLocal(particion)

            si peso(nuevaParticion) < pesoMin {
                pesoMin = peso(nuevaParticion)
                particion = nuevaParticion
            }
        }
    } sino criterioDeTerminacion = CUANDO_EL_MEJOR_SE_REPITA_X_VECES {
        nuevaParticion = funcionGoloso(grafo, cantidadDeConjuntos)
        funcionLocal(nuevaParticion)
        contadorRepeticiones = 1

        si peso(nuevaParticion) < pesoMin {
            pesoMin = peso(nuevaParticion)
            particion = nuevaParticion
            contadorRepeticiones = 1
        } sino peso(nuevaParticion = pesoMin) {
            contadorRepeticiones += 1
        }
    }

    return particion
}
\end{verbatim}



\newpage
\subsection{Experimentación y gráficos.}
\vspace*{0.3cm}
\textcolor{red}{\textbf{completar!}}
