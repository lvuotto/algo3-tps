\documentclass[a4paper]{article}
\usepackage[spanish]{babel}
\usepackage[utf8]{inputenc}
\usepackage{fancyhdr}
\usepackage{charter}   % tipografia
\usepackage{graphicx}
\usepackage{makeidx}

\usepackage{float}
\usepackage{amsmath, amsthm, amssymb}
\usepackage{amsfonts}
\usepackage{sectsty}
\usepackage{wrapfig}
\usepackage{listings}
\usepackage{caption}

\usepackage{hyperref} %las entradas del índice tienen links
\hypersetup{
    colorlinks=true,
    linktoc=all,
    citecolor=black,
    filecolor=black,
    linkcolor=black,
    urlcolor=black
}

\usepackage{color} % para snipets de codigo coloreados
\usepackage{fancybox}  % para el sbox de los snipets de codigo

\definecolor{litegrey}{gray}{0.94}

% \newenvironment{sidebar}{%
% 	\begin{Sbox}\begin{minipage}{.85\textwidth}}%
% 	{\end{minipage}\end{Sbox}%
% 		\begin{center}\setlength{\fboxsep}{6pt}%
% 		\shadowbox{\TheSbox}\end{center}}
% \newenvironment{warning}{%
% 	\begin{Sbox}\begin{minipage}{.85\textwidth}\sffamily\lite\small\RaggedRight}%
% 	{\end{minipage}\end{Sbox}%
% 		\begin{center}\setlength{\fboxsep}{6pt}%
% 		\colorbox{litegrey}{\TheSbox}\end{center}}

\newenvironment{codesnippet}{%
	\begin{Sbox}\begin{minipage}{\textwidth}\sffamily\small}%
	{\end{minipage}\end{Sbox}%
		\begin{center}%
		\colorbox{litegrey}{\TheSbox}\end{center}}



\input{page.layout}
\usepackage{underscore}
\usepackage{caratula}
\usepackage{url}

\usepackage{color}
\usepackage{clrscode3e} % para el pseudocodigo




\begin{document}

\lstset{
  language=C++,
  backgroundcolor=\color{white},   % choose the background color
  basicstyle=\footnotesize,        % size of fonts used for the code
  breaklines=true,                 % automatic line breaking only at whitespace
  captionpos=b,                    % sets the caption-position to bottom
  commentstyle=\color{mygreen},    % comment style
  escapeinside={\%*}{*)},          % if you want to add LaTeX within your code
  keywordstyle=\color{blue},       % keyword style
  stringstyle=\color{mymauve},     % string literal style
}

\thispagestyle{empty}
\materia{Algoritmos y Estructuras de Datos III}
\submateria{Segundo Cuatrimestre de 2014}
\titulo{Trabajo Práctico III}
%\subtitulo{}
\integrante{González Alba, Pablo}{476/10}{pablo.gonzalez.alba@gmail.com}
\integrante{Quiroz, Nicolás}{450/11}{nquiroz@dc.uba.ar}
\integrante{Vuotto, Lucas}{385/12}{lvuotto@dc.uba.ar}

\maketitle
\newpage

\thispagestyle{empty}
\vfill
\begin{abstract}
    \vspace{0.5cm}

    En este trabajo práctico, resolveremos un problema (k-PMP) para el cual no se conoce un algoritmo polinomial que lo resuelva. Para esto, implementamos una solución exacta que resuelva el problema en tiempo de ejecución exponencial y otras soluciones aproximadas utilizando
    \textit{heurísticas}, que no resuelven el problema con exactitud, sino con mayor o menor grado de precisión, pero cuyo tiempo de ejecución es polinomial y son más viables en la práctica.

    Finalmente pondremos nuestros algoritmos a prueba en \textbf{diferentes escenarios}, ilustrando con gráficos representativos y sacando
    conclusiones en base a los resultados obtenidos.
\end{abstract}

\thispagestyle{empty}
\vspace{1.5cm}
\tableofcontents
\newpage



%\normalsize
\newpage
\section{Objetivos generales}
\textcolor{red}{\textbf{completar!}}



\newpage
\section{Plataforma de pruebas}
El testeo de los algoritmos implementados fue realizado, principalmente, en las máquinas del laboratorio 3 del DC. \newline
\begin{itemize}
  \item \textbf{Sistema Operativo:} Ubuntu Linux 12.04 x86_64, kernel 3.2.0-30-generic

  \item \textbf{Especificaciones del Software:} el código está implementado en \textbf{C++}, compilado con \verb|-std=c++0x|.
  Utilizamos \textbf{Bash} y \textbf{Ruby} para los scripts. Los gráficos fueron realizados con \textbf{gnuplot}.

  \item \textbf{Especificaciones del Hardware:} Intel(R) Core(TM) i5-2500K CPU @ 3.30GHz, 8GB de RAM.
\end{itemize}



\newpage
\section{Ejercicio 1: Sobre k-PMP}
\subsection{Relación entre k-PMP y el problema 3 del TP 1.}
\vspace*{0.3cm}

Una relación posible entre k-PMP y el problema 3 del TP1 podría ser que, si
k-PMP tiene una solución, de manera tal que el peso total de la $k$-partición sea menor o igual al umbral M de peligrosidad del P3 del TP1, esto significa que el P3 del TP1 tiene solución con k camiones, aunque no necesariamente sea la solución óptima a este problema.

Es decir, quizás el P3 del TP1 pueda resolverse utilizando menos de $k$ camiones, pero no podemos inferir esta información a partir del resultado del
k-PMP, sólo saber si funciona para k.


\newpage
\subsection{Relación entre k-PMP y el problema de coloreo de los vértices de
            un grafo.}
\vspace*{0.3cm}

Sea $G = (V, E)$ un grafo simple. Dado un entero $k$ y una función de peso $w$
que, dado un eje $e \in E$ le asigna a $e$ un peso \textbf{estrictamente positivo},
la relación entre el problema de $k-PMP$ y el problema de coloreo de los vértices
de un grafo puede resumirse en la ecuación siguiente:

\begin{center}
  k-PMP($G$) = $0 \iff G$ es $k$-coloreable
\end{center}

Esto vale pues, si la $k$-partición de $G$ de \textbf{peso mínimo} hallada
tiene peso $0$, significa que cada conjunto de la partición (subconjuntos de $V$)
no posee aristas \textit{intrapartición}, es decir, sus extremos no se encuentran en
un mismo conjunto, por lo tanto el peso de cada uno de estos conjuntos será
\textbf{nulo} y en consecuencia también lo será la suma total de los pesos de
las aristas \textit{intrapartición}. Si esto sucede, entonces el grafo $G$ es
$k$-coloreable, pues esto significa que $V$ puede dividirse en, a lo sumo, $k$
conjuntos, siendo $k$ la cantidad máxima de colores a utilizar, de manera que no
haya vértices adyacentes en estos conjuntos, pues el \textit{coloreo} no nos permite
utilizar colores iguales para vértices adyacentes.

De la misma forma, si el grafo $G$ es $k$-coloreable, significa que utilizando
como máximo $k$ colores podemos $colorear$ los vértices del mismo, esto implica
que $V \in G$ puede dividirse en $i$ conjuntos, con $i \in {1, \dots, k}$, siendo $i$
la cantidad de colores utilizada para colorear los vértices del grafo, de manera
que en cada uno de estos conjuntos no haya vértices adyacentes, pues los agrupamos
según el color que le fue asignado a cada vértice. Esto implica que el peso de
las aristas \textit{intrapartición} es nulo para todos los $i$ conjuntos, pues éstas no
existen, ya que las aritas presentes en $G$ no conectan vértices pertenecientes a
un mismo subconjunto de $V$. Por lo tanto, al ser el peso del conjunto $i$ nulo,
la suma de los pesos de las aristas \textit{intrapartición} será también nulo, por lo
que el peso de la $k$-partición de $G$ será nulo, por lo tanto vale que $k-PMP(G) = 0$.

Es importante destacar que la relación planteada anteriormente y expresada mediante
la ecuación descripta más arriba sólo vale si la función de peso $w$ asigna a las
aristas valores positivos, caso contrario la partición peso mínimo podría ser una
cuyo peso sea negativo, la cantidad de aristas \textit{intrapartición} podría variar y el
\textit{coloreo} del grafo $G$ necesitaría más de $k$ colores para llevarse a cabo.



\newpage
\subsection{Situaciones de la vida real que pueden modelarse con k-PMP.}
\vspace*{0.3cm}
\textcolor{red}{\textbf{completar!}}



\newpage
\section{Ejercicio 2: Algoritmo exacto para k-PMP}
\subsection{Descripción del algoritmo implementado. Podas y estrategias.}
\vspace*{0.3cm}

\textcolor{red}{\textbf{completar!}}

El algoritmo es un backtracking que recorre todos los nodos y todos los
conjuntos posibles y, recursivamente, se fija cual es mejor.  La única poda
consiste en cada iteración calcular si el peso actual de la partición es mayor
al peso de la partición mínima encontrada hasta ese momento y, de ser así,
cortar la recursión.


\newpage

\subsection{Análisis de complejidad en el peor caso.}
\vspace*{0.3cm}

En primer lugar, es necesario observar que, si $k > n$, la solución el problema
es distribuir los vértices en $n$ particiones, es decir, poner un vértice en
cada partición, obteniendo peso $0$ en total, pues no hay aristas
intrapartición. Luego, al no haber diferencia en la cantidad en la resolución
del problema si $k > n$ a si $k = n$ (salvo por la presencia de particiones
vacías), se asume $k \le n$.

Teniendo en cuenta esta observación, el algoritmo busca encontrar la
partición en $k$ subconjuntos \textit{no vacíos} los $n$ vértices del grafo,
asegurando que la sumatoria de los pesos de cada uno de los $k$ subconjuntos
sea mínima. Esto conlleva, en el peor de los casos, a buscar todos los posibles
modos de distribuir $n$ elementos distinguibles en $k$ conjunto distinguibles,
de modo tal que no haya elementos repetidos en los $k$ conjuntos. La cantidad
de particiones generadas coincide en este caso con el número de
Stirling\footnote{
\url{https://en.wikipedia.org/wiki/Stirling_numbers_of_the_second_kind}} $S(n,
k)$.

Esto se logra insertando de a uno los vértices del grafo en algún conjunto de
la $k$-partición. Por cada uno de los $k$ conjuntos, el procedimiento es el
siguiente:
\begin{itemize}
  \item se inserta en el conjunto, lo que toma $O(n)$, pues para insertar un
  vértice en un conjunto de la partición, se debe calcular el nuevo peso de
  dicho conjunto y en el peor de los casos, todos los vértices pertenecen al
  conjunto.
  \item se comprueba si quedan no quedan más vértices por agregar y, de ser
  así, si el peso de la partición actual es menor al peso de la mejor partición
  encontrada hasta el momento, se toma la partición actual como la mejor
  partición hasta el momento, realizando una copia de dicha partición, lo que
  toma $O(n)$, pues en el peor de los casos todos los vértices están
  distribuidos en los conjuntos de la $k$-partición.
  \item en caso contrario, se sigue con la recursión.
\end{itemize}

Luego, en la peor de las situaciones, la complejidad de cada paso de la
recursión es $O(n . k)$.

Esto se realiza hasta generar todas las posibles particiones en $k$ conjuntos.
Luego, la complejidad total del algoritmo es
\begin{align*}
  O(S(n, k) . n . k)
\end{align*}


\newpage \subsection{Experimentación y gráficos.}
\vspace*{0.3cm}

\subsubsection{Test algoritmo exacto}

(ver \verb|info.exacto.dat.promedio|) \medskip

Para realizar este test, se generaron aleatoriamente grafos de $n$ nodos, con  $n$ inicializado en 5 e incrementándose de a 1, hasta alcanzar 30 y $m$
(cantidad de aristas) vale $\frac{n^2}{5}$, $k$ vale $\frac{n}{3}$

Para cada instancia se toma el \textbf{valor mínimo}, medido en microsegundos, luego de \textbf{10 corridas}.

Dada una combinación de $m$, $n$ y $k$, se generaron 5 instancias aleatorias con dicha combinación y se consideró el promedio entre ellas.

\vspace*{0.5cm}

\begin{figure}[h]
  \begin{center}
    \includegraphics[scale=0.35]{imagenes/grafico-exacto.png}
  \end{center}
\end{figure}

\vspace*{0.5cm}

En el gráfico puede apreciarse que el algoritmo es \textit{``lento''}, lo cual era esperado debido a su naturaleza exponencial.


\newpage
\section{Ejercicio 3: Heurística constructiva golosa para k-PMP}
\subsection{Descripción del algoritmo implementado.}
\vspace*{0.3cm}

\textcolor{red}{\textbf{completar!}}

Con la idea de que evitando las aristas más pesadas se puede reducir el peso
total, el algoritmo consiste en ordenar las aristas por peso en orden
descendente. Luego, por cada arista, se agregan sus 2 nodos a los conjuntos que
resulten en el menor peso total.

Pseudo:
goloso(grafo, cantidadDeConjuntos) {
	particion = Particion(grafo, cantidadDeConjuntos)

	aristas = ordenarPorPeso(aristas(grafo))

	por cada arista en aristas {
		agregarAlDeMenosPeso(particion, vertice1(arista))
		agregarAlDeMenosPeso(particion, vertice2(arista))
	}

	return particion
}

agregarAlDeMenosPeso(particion, vertice) {
	si contiene(particion, vertice)
		return

	minConjunto = primero(conjuntos(particion))
	minCosto = costo(minConjunto, vertice)

	por cada conjunto en conjuntos(particion) {
		si costo(conjunto, vertice) < minCosto {
			minConjunto = conjunto
			minCosto = costo(conjunto, vertice)
		}
	}
}

costo(conjunto, nuevoVertice) {
	costo = 0
	por cada vertice en conjunto {
		costo += costo(vertice, nuevoVertice)
	}

	return costo
}


\newpage
\subsection{Análisis de complejidad en el peor caso.}
\vspace*{0.3cm}

\textcolor{red}{\textbf{completar!}}

n = |vertices|
m = |aristas|
k = |conjuntos|
Se ordenan las aristas. O(m * log(m)).
Se recorren todas y por cada una se calcula el costo de agregarla a cada
conjunto. Calcular el costo en un conjunto es O(n), porque se comparan todos
los vértices, lo que llevaría a pensar que es O(k * n), pero como cada
conjunto tiene vértices distintos que los otros, en total nunca se hacen más
de n comparaciones de vértices. Es decir, O(k + n).

Total: O(m * log(m) + m * (k + n))

(Todas las otras cosas que se hacen son O(1))



\newpage
\subsection{Instancias de k-PMP para las cuales la heurística no proporciona
            una solución óptima.}
\vspace*{0.3cm}
\textcolor{red}{\textbf{completar!}}

Como el algoritmo intenta deshacerse primero de las aristas más pesadas, el
peor caso es cuando el problema lo dan muchas aristas de poco peso.

Esta situación se puede generar de la siguiente forma: Dados k conjuntos, el
grafo debe tener 2 tipos de vértices: k vértices, todos ellos adyacentes entre
sí con un peso C > 1, y luego n vértices, cada uno adyacente únicamente a los
primeros k vértices con un peso C', 0 < C' < C.  Por como se comporta el
algoritmo, las aristas que unen a los primeros k vértices se ubicarán primero,
dando que queden los k vértices en conjuntos distintos.  Luego, de cualquier
forma que se agreguen los siguientes vértices, cada uno generará un costo  C',
siendo el peso total n*C'.  Sin embargo, poniendo 2 de los primeros k vértices
juntos en un mismo conjunto, se genera un costo C, pero todos los n vértices
pueden entrar en un mismo conjunto sin aumentar el costo.  De esta forma, con C
y C' fijos, n*C' > C a partir de algún n, por lo que se puede generar un grafo
tan malo como se quiera.


\newpage \subsection{Experimentación y gráficos.}
\vspace*{0.3cm}

\textcolor{red}{\textbf{completar!}}


\newpage
\section{Ejercicio 4: Heurística de búsqueda local para k-PMP}
\subsection{Descripción del algoritmo implementado.}
\vspace*{0.3cm}

La \textbf{heurística de búsqueda local} consiste en una heurística que comienza con una solución dada que intentamos mejorar. Llamaremos a esta solución ``candidata''. Luego, revisamos iterativamente sus soluciones ``vecinas''. Este conjunto de soluciones vecinas conforma el espacio de búsqueda y sus elementos son también potenciales soluciones candidatas. Esto sólo es posible si la vecindad contiene más elementos aparte de nuestra solución actual.

Si existe una mejor solución, se toma como solución actual y se repite el proceso, buscando en la vecindad de la nueva solución.

En particular, utilizamos una solución generada por la heurística golosa y verificamos mediante la búsqueda local si ésta es mejorable.

\vspace*{0.3cm}

Para nuestro algoritmo, planteamos las siguientes 2 estrategias para elegir vecindades:

\begin{itemize}
    \item \textbf{mover:} Una partición es vecina de otra si consiste en mover un único vértice de un conjunto a otro, es decir, $P$ y $Q$ son vecinas si existen $v \in A \in P$ y $B \in Q$ tales que $A - v = B$.

    \item \textbf{intercambiar:} Una partición es vecina de otra si consiste en intercambiar 2 vértices de distintos conjuntos, es decir, $P$ y $Q$ son vecinas si existen $v \in A \in P$ y $w \in B \in Q$ tales que $(A - v) \cup \{w\} = (B - w) \cup \{v\}$.
\end{itemize}

Los algoritmos consisten en, partiendo de una solución inicial, probar todas las soluciones vecinas, y continuar desde la de menor peso. Cuando no se pueda mejorar más, finaliza la ejecución.

\vspace*{0.5cm}

\textbf{Pseudocódigo con la estrategia mover:}

\vspace*{0.3cm}

\begin{verbatim}
kpmp_mover(particion) {
    pesoMin = peso(particion)
    sePuedeMejorar = true
    mientras sePuedeMejorar {
        por cada vertice en vertices(particion) {
            conjuntoDelVertice = buscarConjunto(particion, vertice)
            pesoSinVertice = peso(particion) - costo(conjuntoDelVertice, vertice)
            por cada conjunto en conjuntos(particion) excepto conjuntoDelVertice {
                peso = pesoSinVertice + costo(conjunto, vertice)
                si peso < pesoMin {
                    pesoMin = peso
                    verticeMin = vertice
                    conjuntoViejo = conjuntoDelVertice
                    conjuntoNuevo = conjunto
                }
            }
        }
        si pesoMin < peso(particion) {
            sacarVertice(conjuntoViejo, verticeMin)
            agregarVertice(conjuntoNuevo, verticeMin)
        } sino {
            sePuedeMejorar = false
        }
    }
}
\end{verbatim}

\newpage

\textbf{Pseudocódigo con la estrategia intercambiar:}

\vspace*{0.3cm}

\begin{verbatim}
kpmp_switch(particion) {
    pesoMin = peso(particion)
    sePuedeMejorar = true
    mientras sePuedeMejorar {
        por cada vertice1 en vertices(particion) {
            conjuntoDelVertice1 = buscarConjunto(particion, vertice)
            por cada vertice2 en vertices(particion) excepto que compartan conjunto {
                conjuntoDelVertice2 = buscarConjunto(particion, vertice)
                peso = peso(particion)
                    - costo(conjuntoDelVertice1, vertice1)
                    - costo(conjuntoDelVertice2, vertice2)
                    + costo(conjuntoDelVertice1, vertice2)
                    + costo(conjuntoDelVertice2, vertice1)
                si peso < pesoMin {
                    pesoMin = peso
                    verticeMin1 = vertice1
                    verticeMin2 = vertice2
                    conjuntoMin1 = conjuntoDelVertice1
                    conjuntoMin2 = conjuntoDelVertice2
                }
            }
        }
        si pesoMin < peso(particion) {
            sacarVertice(conjuntoDelVertice1, verticeMin1)
            agregarVertice(conjuntoDelVertice1, verticeMin2)
            sacarVertice(conjuntoDelVertice2, verticeMin2)
            agregarVertice(conjuntoDelVertice2, verticeMin1)
        } sino {
            sePuedeMejorar = false
        }
    }
}
\end{verbatim}



\newpage
\subsection{Análisis de complejidad del peor caso de una iteración del
            algoritmo de búsqueda local.}
\vspace*{0.3cm}

Sea $G = (V,E)$ y consideremos $n = |V|$ y $m = |E|$.

La función \texttt{buscarConjunto} recorre todos los conjuntos y en cada uno
busca, en tiempo logarítmico, el vértice dado. Esta función realiza sobre cada
conjunto $O(\log(\text{cantidad de vértices del conjunto}))$ comparaciones,
realizadas en $O(1)$. A su vez, la cantidad de elementos de cada conjunto está
acotada por $n$. Luego, la complejidad de esta función es $O(k\log(n))$, siendo
$k$ el parámetro de entrada.

\vspace*{0.3cm}

En cada iteración, el algoritmo de búsqueda local con la estrategia de
\textit{mover} comienza a recorrer cada vértice del grafo, realizando lo siguiente:
\begin{itemize}
  \item Se llama a la función \texttt{buscarConjunto}, la cual pertenece a
  $O(k\log(n))$.

  \item Por cada conjunto de la partición, se calcula el peso obtenido al
  agregarlo al conjunto, llamando a la función \texttt{costo}. Al igual que en
  la función \texttt{agregarAlDeMenosPeso} explicada en el ítem anterior, esto
  pertenece a $O(k + n)$.
\end{itemize}

Luego de recorrer cada vértice, se pueden llegar a realizar 2 llamados a
la función \texttt{costo}, que pertenece a $O(n)$.

Por lo tanto, en cada iteración del algoritmo de busqueda local, la complejidad
es $O(n (k\log(n) + k + n) + n) = O(nk\log(n) + nk + n^2 + n) = O(nk\log(n) +
n^2)$.

\vspace*{0.3cm}

En cada iteración, el algoritmo de búsqueda local con la estrategia de
intercambiar comienza a recorrer cada vértice del grafo, realizando lo siguiente:

\begin{itemize}
  \item Se llama a la función \texttt{buscarConjunto}, la cual pertenece a
  $O(k\log(n))$.

  \item Por cada uno de los otros vértices que no están en el mismo conjunto
  que el vértice actual:

  \begin{itemize}
    \item Se llama de nuevo a la función \texttt{buscarConjunto}.

    \item Se llama 4 veces a la función \texttt{costo}.
  \end{itemize}
\end{itemize}

Finalmente, se pueden llegar a realizar 4 llamados más a la función
\texttt{costo}.

Por lo tanto, en cada iteración del algoritmo de búsqueda local, la complejidad
es $O(n (k\log(n) + n (k\log(n) + k + n)) + k + n)$. Desarrollando, la
complejidad del algoritmo pertenece a $O(n^2k\log(n) + n^3)$.


\newpage \subsection{Experimentación y gráficos.}

\subsubsection{Tiempos de ejecución}

Para comparar los tiempos de ejecución generamos diversos grafos de entrada
variando sus tamaños y, además, probamos con una solución inicial completamente
random y con una generada por el algoritmo goloso.
\\

Variando la cantidad de vértices de un grafo denso (m = 3/4 n), a partir de la
solución que devuelve el algoritmo goloso. N va de 20 en 20.
\vspace*{0.5cm}

\begin{figure}[h]
  \begin{center}
    \includegraphics[scale=0.35]{imagenes/local-goloso-n-tiempo.png}
  \end{center}
\end{figure}

\vspace*{0.5cm}

Variando la cantidad de vértices de un grafo denso (m = 3/4 n), a partir de una
solución al azar. N va de 20 en 20.
\vspace*{0.5cm}

\begin{figure}[h]
  \begin{center}
    \includegraphics[scale=0.35]{imagenes/local-random-n-tiempo.png}
  \end{center}
\end{figure}

\vspace*{0.5cm}

Variando la cantidad de conjuntos de un grafo denso (n = 200, m = 150), a partir
de la solución que devuelve el algoritmo goloso. K aumenta de a 1.
\vspace*{0.5cm}

\begin{figure}[h]
  \begin{center}
    \includegraphics[scale=0.35]{imagenes/local-goloso-k-tiempo.png}
  \end{center}
\end{figure}

\vspace*{0.5cm}

Variando la cantidad de conjuntos de un grafo denso (n = 200, m = 150), a partir
de una solución al azar. K aumenta de a 1.
\vspace*{0.5cm}

\begin{figure}[h]
  \begin{center}
    \includegraphics[scale=0.35]{imagenes/local-random-k-tiempo.png}
  \end{center}
\end{figure}

\vspace*{0.5cm}


Como puede observarse, en todos los casos el algortimo de $intercambiar$ es más
lento que el algoritmo de $mover$.

Algo que era esperable considerando que $intercambiar$ en cada iteración tiene
orden $O(n^2k\log(n) + n^3)$ vs el $O(nk\log(n) + n^2)$ de $mover$.

También se puede apreciar que el tiempo crece (en ambos casos) cuando aumenta
el $n$, también razonable ya que la complejidad depende de éste.

Sin embargo, aunque podría esperarse que también aumente ligado al $k$, con
una solución al azar, el algoritmo $mover$ se mantiene constante y el
algoritmo $intercambiar$ varía mucho, aunque también acotado. No sucede lo
mismo cuando se comienza con la solución del goloso, que se comporta
invérsamente a $k$.

Esto seguramente se deba a que, con muchos conjuntos disponible, la heurísitica
golosa da una solución muy cercana a la ideal. Luego, las heurísiticas de
búsqueda local, necesitan ejecutarse pocas veces hasta encontrar un resultado
que no puedan mejorar

\newpage \subsubsection{Calidad}

Para comparar la calidad, observamos los pesos totales de la partición
utilizando las mismas variables que para comparar los tiempos de ejecución.

Un peso menor indica una solución de mejor calidad, queremos ver si alguna
de las dos heurísticas tiene una mejor calidad consistentemente.
\\

Variando la cantidad de vértices de un grafo denso (m = 3/4 n), a partir de la
solución que devuelve el algoritmo goloso. N va de 20 en 20.
\vspace*{0.5cm}

\begin{figure}[h]
  \begin{center}
    \includegraphics[scale=0.35]{imagenes/local-goloso-n-peso.png}
  \end{center}
\end{figure}

\vspace*{0.5cm}

Variando la cantidad de vértices de un grafo denso (m = 3/4 n), a partir de una
solución al azar. N va de 20 en 20.
\vspace*{0.5cm}

\begin{figure}[h]
  \begin{center}
    \includegraphics[scale=0.35]{imagenes/local-random-n-peso.png}
  \end{center}
\end{figure}

\vspace*{0.5cm}

Variando la cantidad de conjuntos de un grafo denso (n = 200, m = 150), a partir
de la solución que devuelve el algoritmo goloso. K aumenta de a 1.
\vspace*{0.5cm}

\begin{figure}[h]
  \begin{center}
    \includegraphics[scale=0.35]{imagenes/local-goloso-k-peso.png}
  \end{center}
\end{figure}

\vspace*{0.5cm}

Variando la cantidad de conjuntos de un grafo denso (n = 200, m = 150), a partir
de una solución al azar. K aumenta de a 1.
\vspace*{0.5cm}

\begin{figure}[h]
  \begin{center}
    \includegraphics[scale=0.35]{imagenes/local-random-k-peso.png}
  \end{center}
\end{figure}

\vspace*{0.5cm}

Con estos gráficos se puede observar que el algoritmo $mover$ genera resultados
con mejor calidad que el algoritmo $intercambiar$ para casi todos los casos.

Variando la cantidad de vértices, el algoritmo $mover$ da resultados muy similares
empezando con la solución al azar tanto como con la solución generada por el
algoritmo goloso. Sin embargo, el algoritmo $intercambiar$ encuentra una solución
de calidad similar únicamente comenzando desde la solución del goloso.
Empezando desde la solución al azar, el peso crece mucho más rápido que el peso
del algoritmo $mover$.

Algo similar encontramos al variar la cantidad de conjuntos. Cuando se comienza
desde una solución del goloso, se comportan de forma similar. Pero empezando con
una solución aleatoria, el algoritmo $intercambiar$ no encuentra buenas soluciones
con la misma consistencia que el algoritmo $mover$.

\\
Luego de este análisis, con mejores tiempos y mejor calidad, queda muy evidente
que conviene utilizar la heurística de búsqueda local $mover$.

\newpage
\section{Ejercicio 5: GRASP}
\subsection{Descripción del algoritmo implementado.}
\vspace*{0.3cm}

El algoritmo de \textbf{GRASP} consiste en utilizar una
\textit{heurística golosa aleatorizada} para obtener una primera solución
y luego ir mejorándola aplicándole una \textit{heurística de búsqueda local}.

Esto se repite hasta llegar a un \textit{criterio de terminación} dado,
quedándose con la mejor solución encontrada hasta el momento.

El algoritmo que diseñamos utiliza el algoritmo goloso planteado en el
ejercicio 3 pero, para la selección de candidatos en la \textit{heurística golosa aleatorizada}, se plantean 2 alternativas:
\begin{enumerate}
\item En lugar de elegir la arista más pesada, se elige una al azar de las $X$ más pesadas.

\item En lugar de poner el vértice en el conjunto que genere el menor peso, se elige un conjunto al azar de los $X$ que generan menor peso.
\end{enumerate}

Luego se mejora con alguno de los algoritmos de búsqueda local planteados
en el ejercicio 4.

Y, finalmente, se diseñaron 2 criterios de terminación:
\begin{enumerate}
\item Elegir la mejor solución luego de $X$ iteraciones.

\item Elegir la mejor solución luego de que éste se encuentre como solución $X$ veces. Si se encuentra una solución mejor, se resetea el contador.
\end{enumerate}


Se probó con cada estrategia de la \textit{heurística golosa aleatorizada} por separado y luego con ambas.

Todos estos resultados fueron probados con ambos algoritmos de localidad.

Por último también se combinaron todos con los 2 criterios de terminación, con distintos valores para las variables $X$ (independientes) detalladas anteriormente.

\vspace*{0.35cm}

\textbf{Pseudocódigo del algoritmo GRASP:}

\vspace*{0.1cm}

\begin{verbatim}
GRASP:
  Hasta que se cumpla el criterio de terminación hacer:
    particion = heurísitica golosa aleatorizada
    mejorar partición con una heurística de búsqueda local
  Devolver la mejor partición hallada

El criterio de terminación puede ser:
- Ejecutar X veces
- Repetir hasta que se encuentre la misma (mejor) solución X veces
\end{verbatim}

\newpage
\subsection{Experimentación y gráficos.}
\vspace*{0.3cm}



\newpage
\section{Ejercicio 6: Experimentación general y comparativa de todos los
         métodos implementados}
\subsection{Implementación elegida}

Elegimos utilizar la meta-heurística GRASP con el algoritmo goloso aleatorizado
en la elección de aristas, eligiendo al azar una de las 10 más pesadas en cada
iteración.

Luego aplicándole el algoritmo de búsqueda local que denominamos $mover$, que
define como vecindades a todas las particiones que difieren de la actual por
tener un único vértice en un conjunto distinto.

Finalmente veremos los dos criterios de terminación:
\begin{itemize}
  \item repitiendo hasta que la mejor solución se encuentre 34 veces.
  \item repitiendo exactamente 47 veces y quedándonos con la mejor hasta el momento.
\end{itemize}

\subsection{Otras implementaciones para contrastar}

Vamos a contrastar los resultados contra las siguientes implementaciones:

\begin{itemize}
  \item Exacto (mientras el tiempo de ejecución sea menor a 5 minutos)

  \item Goloso sin aleatorizar

  \item GRASP, pero aleatorizando tanto aristas como conjuntos

  \item GRASP, pero utilizando el algoritmo de búsqueda local $intercambiar$
\end{itemize}

\subsection{Experimentación}

Experimentamos con distintos tipos de grafos, variando la cantidad de vértices,
variando la cantidad de aristas según el grafo, con dos cantidades de conjuntos
distintas y, para cada combinación de valores, generar 15 grafos distintos.

Analicemos primero los tiempos:

Arbol, k=3
\vspace*{0.5cm}

\begin{figure}[h]
  \begin{center}
    \includegraphics[scale=0.35]{imagenes/ej6-arbol-k3-tiempo.png}
  \end{center}
\end{figure}

\vspace*{0.5cm}

Arbol, k=7
\vspace*{0.5cm}

\begin{figure}[h]
  \begin{center}
    \includegraphics[scale=0.35]{imagenes/ej6-arbol-k7-tiempo.png}
  \end{center}
\end{figure}

\vspace*{0.5cm}

Completo, k=3
\vspace*{0.5cm}

\begin{figure}[h]
  \begin{center}
    \includegraphics[scale=0.35]{imagenes/ej6-completo-k3-tiempo.png}
  \end{center}
\end{figure}

\vspace*{0.5cm}

Completo, k=7
\vspace*{0.5cm}

\begin{figure}[h]
  \begin{center}
    \includegraphics[scale=0.35]{imagenes/ej6-completo-k7-tiempo.png}
  \end{center}
\end{figure}

\vspace*{0.5cm}

Denso, aristas con pesos iguales, k=3
\vspace*{0.5cm}

\begin{figure}[h]
  \begin{center}
    \includegraphics[scale=0.35]{imagenes/ej6-denso-pesos-iguales-k3-tiempo.png}
  \end{center}
\end{figure}

\vspace*{0.5cm}

Denso, aristas con pesos iguales, k=7
\vspace*{0.5cm}

\begin{figure}[h]
  \begin{center}
    \includegraphics[scale=0.35]{imagenes/ej6-denso-pesos-iguales-k7-tiempo.png}
  \end{center}
\end{figure}

\vspace*{0.5cm}

Denso, aristas con pesos distintos, k=3
\vspace*{0.5cm}

\begin{figure}[h]
  \begin{center}
    \includegraphics[scale=0.35]{imagenes/ej6-denso-pesos-distintos-k3-tiempo.png}
  \end{center}
\end{figure}

\vspace*{0.5cm}

Denso, aristas con pesos distintos, k=7
\vspace*{0.5cm}

\begin{figure}[h]
  \begin{center}
    \includegraphics[scale=0.35]{imagenes/ej6-denso-pesos-distintos-k7-tiempo.png}
  \end{center}
\end{figure}

\vspace*{0.5cm}

Aunque es difícil de graficar bien, el exacto crece mucho más rápido que todos
los otros algoritmos.
Lo ejecutamos hasta n=18, pero para que se puedan ver el resto de las líneas,
a veces, lo cortamos incluso antes, como puede verse en el gráfico del grafo
Completo con $k = 7$.

Eso puede hacer aparentar, en algunos casos (como en el gráfico de Arbol con
$k = 3$), que el exacto crece más lento, pero en realidad es que si agregáramos
uno o más puntos, crecería tanto que el resto de las líneas parecerían rectas
sobre el 0.

Lo primero que se puede notar, es que el goloso sin aleatorizar es el más
rápido, algo completamente esperable, ya que se ejecuta sólamente una vez contra
la ejecución de la heurística local y repetida decenas de veces en el GRASP.

Luego, también se observa que el peor de los GRASP es que utiliza la heurística
local de $intercambiar$, seguido del algoritmo GRASP que aleatoriza tanto
aristas como conjuntos.

Los dos algoritmos elegidos son los que menor tiempo tardan (exceptuando el
goloso puro), con una pequeña ventaja el que realiza menos repeticiones.

Sin embargo, todavía falta contrastar la calidad de estos:

Arbol, k=3
\vspace*{0.5cm}

\begin{figure}[h]
  \begin{center}
    \includegraphics[scale=0.35]{imagenes/ej6-arbol-k3-peso.png}
  \end{center}
\end{figure}

\vspace*{0.5cm}

Completo, k=3
\vspace*{0.5cm}

\begin{figure}[h]
  \begin{center}
    \includegraphics[scale=0.35]{imagenes/ej6-completo-k3-peso.png}
  \end{center}
\end{figure}

\vspace*{0.5cm}

Completo, k=7
\vspace*{0.5cm}

\begin{figure}[h]
  \begin{center}
    \includegraphics[scale=0.35]{imagenes/ej6-completo-k7-peso.png}
  \end{center}
\end{figure}

\vspace*{0.5cm}

Denso, aristas con pesos iguales, k=3
\vspace*{0.5cm}

\begin{figure}[h]
  \begin{center}
    \includegraphics[scale=0.35]{imagenes/ej6-denso-pesos-iguales-k3-peso.png}
  \end{center}
\end{figure}

\vspace*{0.5cm}

Denso, aristas con pesos iguales, k=7
\vspace*{0.5cm}

\begin{figure}[h]
  \begin{center}
    \includegraphics[scale=0.35]{imagenes/ej6-denso-pesos-iguales-k7-peso.png}
  \end{center}
\end{figure}

\vspace*{0.5cm}

Denso, aristas con pesos distintos, k=3
\vspace*{0.5cm}

\begin{figure}[h]
  \begin{center}
    \includegraphics[scale=0.35]{imagenes/ej6-denso-pesos-distintos-k3-peso.png}
  \end{center}
\end{figure}

\vspace*{0.5cm}

Denso, aristas con pesos distintos, k=7
\vspace*{0.5cm}

\begin{figure}[h]
  \begin{center}
    \includegraphics[scale=0.35]{imagenes/ej6-denso-pesos-distintos-k7-peso.png}
  \end{center}
\end{figure}

\vspace*{0.5cm}

Lo primero que podemos notar, es que en el árbol (un grafo bipartito, que puede
dividirse en 2 conjuntos y generar peso 0) y con $k = 3$, no todos llegan al
resultado obvio. Con $k = 7$ (no graficado) todos los algoritmos encuentran la
solución de peso 0.

Lo siguiente que se hace evidente, es que ninguno logra llegar al resultado
exacto (para $n > 10$) y, sin embargo, dan resultados bastante similares,
ninguno crece en mayor medida que los otros.

Finalmente, el peor de todos, igualmente, es el goloso sin aleatorizar,
seguido del GRASP que utiliza la heurística de búsqueda local de $intercambiar$.

Los otros tres algoritmos: los dos elegidos y el GRASP que aleatoriza aristas y
conjuntos, tienen resultados muy parecidos, con diferencias menores al 1\% y sin
que ninguno tenga peso inferior consistentemente sobre los otros dos.

Considerando que el que aleatoriza tanto aristas como conjuntos era más lento
que los otros dos, podemos confirmar que la elección hecha fue la correcta:

Utilizando la meta-heurísitica GRASP, aleatorizando la selección de aristas en
el algoritmo goloso, usando la vecindad $mover$ en el algoritmo de búsqueda
local y sin importar el criterio de terminación, pero dando un número
considerable de repeticiones (34 y 47), obtuvimos los mejores resultados en
todas las pruebas.

Sin embargo, no fueron lo suficientemente buenas como para dar el resultado
exacto (o bastante cercano), por lo que queda abierto a mejoras.

Otro dato interesante es que la heurística golosa sin aleatorizar, aunque es
definitivamente la de peor calidad, la diferencia de peso total no es tan grande,
pero si lo es la diferencia en tiempos de ejecución, hasta 3 órdenes de magnitud
más rápido para grafos densos.






% \newpage
% \section{Apéndice 1: acerca de los tests}
% \textcolor{red}{\textbf{completar!}}



% \newpage
% \section{Apéndice 2: secciones relevantes del código}
% \textcolor{red}{\textbf{completar!}}


\end{document}
