\documentclass[a4paper]{article}
\usepackage[spanish]{babel}
\usepackage[utf8]{inputenc}
\usepackage{fancyhdr}
\usepackage{charter}   % tipografia
\usepackage{graphicx}
\usepackage{makeidx}

\usepackage{float}
\usepackage{amsmath, amsthm, amssymb}
\usepackage{amsfonts}
\usepackage{sectsty}
\usepackage{wrapfig}
\usepackage{listings}
\usepackage{caption}

\usepackage{hyperref} %las entradas del índice tienen links
\hypersetup{
    colorlinks=true,
    linktoc=all,
    citecolor=black,
    filecolor=black,
    linkcolor=black,
    urlcolor=black
}

\usepackage{color} % para snipets de codigo coloreados
\usepackage{fancybox}  % para el sbox de los snipets de codigo

\definecolor{litegrey}{gray}{0.94}

% \newenvironment{sidebar}{%
% 	\begin{Sbox}\begin{minipage}{.85\textwidth}}%
% 	{\end{minipage}\end{Sbox}%
% 		\begin{center}\setlength{\fboxsep}{6pt}%
% 		\shadowbox{\TheSbox}\end{center}}
% \newenvironment{warning}{%
% 	\begin{Sbox}\begin{minipage}{.85\textwidth}\sffamily\lite\small\RaggedRight}%
% 	{\end{minipage}\end{Sbox}%
% 		\begin{center}\setlength{\fboxsep}{6pt}%
% 		\colorbox{litegrey}{\TheSbox}\end{center}}

\newenvironment{codesnippet}{%
	\begin{Sbox}\begin{minipage}{\textwidth}\sffamily\small}%
	{\end{minipage}\end{Sbox}%
		\begin{center}%
		\colorbox{litegrey}{\TheSbox}\end{center}}



\usepackage{fancyhdr}
\pagestyle{fancy}

%\renewcommand{\chaptermark}[1]{\markboth{#1}{}}
\renewcommand{\sectionmark}[1]{\markright{\thesection\ - #1}}

\fancyhf{}

\fancyhead[LO]{Sección \rightmark} % \thesection\
\fancyfoot[LO]{\small{Pablo González Alba, Nicolás Quiroz, Agustín Vaghi, Lucas Vuotto}}
\fancyfoot[RO]{\thepage}
\renewcommand{\headrulewidth}{0.5pt}
\renewcommand{\footrulewidth}{0.5pt}
\setlength{\hoffset}{-0.8in}
\setlength{\textwidth}{16cm}
%\setlength{\hoffset}{-1.1cm}
%\setlength{\textwidth}{16cm}
\setlength{\headsep}{0.5cm}
\setlength{\textheight}{25cm}
\setlength{\voffset}{-0.7in}
\setlength{\headwidth}{\textwidth}
\setlength{\headheight}{13.1pt}

\renewcommand{\baselinestretch}{1.1}  % line spacing


\usepackage{underscore}
\usepackage{caratula}
\usepackage{url}

\usepackage{color}
\usepackage{clrscode3e} % para el pseudocodigo




\begin{document}

\lstset{
  language=C++,
  backgroundcolor=\color{white},   % choose the background color
  basicstyle=\footnotesize,        % size of fonts used for the code
  breaklines=true,                 % automatic line breaking only at whitespace
  captionpos=b,                    % sets the caption-position to bottom
  commentstyle=\color{mygreen},    % comment style
  escapeinside={\%*}{*)},          % if you want to add LaTeX within your code
  keywordstyle=\color{blue},       % keyword style
  stringstyle=\color{mymauve},     % string literal style
}

\thispagestyle{empty}
\materia{Algoritmos y Estructuras de Datos III}
\submateria{Segundo Cuatrimestre de 2014}
\titulo{Trabajo Práctico III}
%\subtitulo{}
\integrante{González Alba, Pablo}{476/10}{pablo.gonzalez.alba@gmail.com}
\integrante{Quiroz, Nicol\'as}{450/11}{nquiroz@dc.uba.ar}
\integrante{Vuotto, Lucas}{385/12}{lvuotto@dc.uba.ar}

\maketitle
\newpage

\thispagestyle{empty}
\vfill
\begin{abstract}
    \vspace{0.5cm}
    \textcolor{red}{\textbf{completar!}}
\end{abstract}

\thispagestyle{empty}
\vspace{1.5cm}
\tableofcontents
\newpage



%\normalsize
\newpage
\section{Objetivos generales}
\textcolor{red}{\textbf{completar!}}



\newpage
\section{Plataforma de pruebas}
El testeo de los algoritmos implementados fue realizado, principalmente, en las máquinas del laboratorio 3 del DC. \newline
\begin{itemize}
  \item \textbf{Sistema Operativo:} Ubuntu Linux 12.04 x86_64, kernel 3.2.0-30-generic

  \item \textbf{Especificaciones del Software:} el código está implementado en \textbf{C++}, compilado con \verb|-std=c++0x|.
  Utilizamos \textbf{Bash} y \textbf{Ruby} para los scripts. Los gráficos fueron realizados con \textbf{gnuplot}.

  \item \textbf{Especificaciones del Hardware:} Intel(R) Core(TM) i5-2500K CPU @ 3.30GHz, 8GB de RAM.
\end{itemize}



\newpage
\section{Ejercicio 1: Sobre k-PMP}
\subsection{a) Relación entre k-PMP y el problema 3 del TP 1.}
\vspace*{0.3cm}
\textcolor{red}{\textbf{completar!}}



\newpage
\subsection{b) Relación entre k-PMP y el problema de coloreo de los vértices de
            un grafo.}
\vspace*{0.3cm}
\textcolor{red}{\textbf{completar!}}



\newpage
\subsection{c) Situaciones de la vida real que pueden modelarse con k-PMP.}
\vspace*{0.3cm}
\textcolor{red}{\textbf{completar!}}



\newpage
\section{Ejercicio 2: Algoritmo exacto para k-PMP}
\subsection{a) Descripción del algoritmo implementado. Podas y estrategias.}
\vspace*{0.3cm}
\textcolor{red}{\textbf{completar!}}



\newpage
\subsection{b) Análisis de complejidad en el peor caso.}
\vspace*{0.3cm}
\textcolor{red}{\textbf{completar!}}



\newpage
\subsection{c) Experimentación y gráficos.}
\vspace*{0.3cm}
\textcolor{red}{\textbf{completar!}}



\newpage
\section{Ejercicio 3: Heurística constructiva golosa para k-PMP}
\subsection{a) Descripción del algoritmo implementado.}
\vspace*{0.3cm}
\textcolor{red}{\textbf{completar!}}



\newpage
\subsection{b) Análisis de complejidad en el peor caso.}
\vspace*{0.3cm}
\textcolor{red}{\textbf{completar!}}



\newpage
\subsection{c) Instancias de k-PMP para las cuales la heurística no proporciona
            una solución óptima.}
\vspace*{0.3cm}
\textcolor{red}{\textbf{completar!}}



\newpage
\subsection{d) Experimentación y gráficos.}
\vspace*{0.3cm}
\textcolor{red}{\textbf{completar!}}


\newpage
\section{Ejercicio 4: Heurística de búsqueda local para k-PMP}
\subsection{a) Descripción del algoritmo implementado.}
\vspace*{0.3cm}
\textcolor{red}{\textbf{completar!}}



\newpage
\subsection{b) Análisis de complejidad del peor caso de una iteración del
            algoritmo de búsqueda local.}
\vspace*{0.3cm}
\textcolor{red}{\textbf{completar!}}



\newpage
\subsection{c) Experimentación y gráficos.}
\vspace*{0.3cm}
\textcolor{red}{\textbf{completar!}}


\newpage
\section{Ejercicio 5: GRASP}
\subsection{a) Descripción del algoritmo implementado.}
\vspace*{0.3cm}
\textcolor{red}{\textbf{completar!}}



\newpage
\subsection{b) Experimentación y gráficos.}
\vspace*{0.3cm}
\textcolor{red}{\textbf{completar!}}


\newpage
\section{Ejercicio 6: Experimentación general y comparativa de todos los
         métodos implementados}
\input{ejercicio6}



\newpage
\section{Apéndice 1: acerca de los tests}
\textcolor{red}{\textbf{completar!}}



\newpage
\section{Apéndice 2: secciones relevantes del código}
\textcolor{red}{\textbf{completar!}}


\end{document}
